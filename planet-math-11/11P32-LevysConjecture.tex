\documentclass[12pt]{article}
\usepackage{pmmeta}
\pmcanonicalname{LevysConjecture}
\pmcreated{2013-03-22 17:26:32}
\pmmodified{2019-11-29}
\pmowner{PrimeFan}{13766}
\pmmodifier{PrimeFan}{13766}
\pmformalizer{chaunguyen}{}
\pmtitle{Levy's conjecture}
\pmrecord{6}{39822}
\pmprivacy{1}
\pmauthor{PrimeFan}{13766}
\pmtype{Conjecture}
\pmcomment{trigger rebuild}
\pmclassification{msc}{11P32}
\pmsynonym{Levy conjecture}{LevysConjecture}
\pmsynonym{Lemoine's conjecture}{LevysConjecture}
\pmdefines{semiprime}

\endmetadata

\usepackage{amssymb}
\usepackage{amsmath}
\usepackage{amsfonts}
\usepackage{cnl}
\usepackage{xcolor}
%\usepackage{graphicx}
%%%%\usepackage{xypic}

\newtheorem{conjecture}{Conjecture}

%- DOCUMENT

\begin{document}
\parskip =\baselineskip
	
% Local Defs 
\def\natdiv#1#2{{#1}\mathrel{|}{#2}}
	
\begin{cnl}
		
\Cnlinput{../Tex2CNL/package/cnlinit}
		
\bigskip
		
%-% BEGIN

\lsection{LevysConjecture}
		
In this section, let $d,\ m,\ k,\ n$ be integers.
		
\dfn{We say that $d$ \df{divides} $m$ iff $d\ne 0$ and there exists an integer $r$ such that $m= d\*r$.
}

\dfn{
We say  $d$ is a \df{divisor} of $m$ iff $d$ divides $m$. 
}

\dfn{
Assume that $p$ is an integer greater than $1$.  Then we say that $p$
is a \df{prime} iff each divisor of $p$ is either equal to $1$ or equal to $p$.
}

\dfn{
We say that $n$
is a \df{semiprime} iff there exist primes $p,\ q$ such that $n=p\*q$.
}

\dfn{
We say that $k$
is an \df{even number} iff there exists an integer $r$ such that $k=2\*r$
}

\dfn{
 We say that $k$
is an \df{odd number} iff there exists an integer $r$ such that $k=2\*r+1$
}

\begin{remark}
Conjecture (\'Emile Lemoine): 
\end{remark}

\begin{conjecture}
If $n$ is an odd integers greater than 5 then there exist an odd prime $p$ and an even semiprime $q$ such that $n=p+q$. 
\end{conjecture}

\begin{remark}
In other words, 
\end{remark}

\begin{conjecture}
If $n$ is an integer greater than 2 then there exist primes $p,\ q$ such that $2n + 1 = p + 2\*q$. 
\end{conjecture}

\begin{remark}

For example, $47 = 13 + 2 \times 17 = 37 + 2 \times 5 = 41 + 2 \times 3 = 43 + 2 \times 2$. A046927 in Sloane's OEIS counts how many different ways $2n + 1$ can be represented as $p + 2q$.

The conjecture was first stated by \'Emile Lemoine in 1894. In 1963, Hyman Levy published a paper mentioning this conjecture in relation to Goldbach's conjecture.
\end{remark}

\end{cnl}



\begin{thebibliography}{3}
\bibitem{ld} L. E. Dickson, {\it History of the Theory of Numbers} Vol. I. Providence, Rhode Island: American Mathematical Society \& Chelsea Publications (1999): 424
\bibitem{rg} R. K. Guy, {\it Unsolved Problems in Number Theory} New York: Springer-Verlag 2004: C1
\bibitem{lh} L. Hodges, ``A lesser-known Goldbach conjecture'', {\it Math. Mag.}, {\bf 66} (1993): 45 - 47. 
\bibitem{el} \'E. Lemoine, ``title'' {\it L'intermediaire des mathematiques} {\bf 179} 3 (1896): 151
\bibitem{hl} H. Levy, ``On Goldbach's Conjecture'', {\it Math. Gaz.} {\bf 47} (1963): 274
\end{thebibliography}

\end{document}