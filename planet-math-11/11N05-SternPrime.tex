\documentclass[12pt]{article}
\usepackage{pmmeta}
\pmcanonicalname{SternPrime}
\pmcreated{2013-03-22 16:19:10}
\pmmodified{2019-11-24}
\pmowner{PrimeFan}{13766}
\pmmodifier{PrimeFan}{13766}
\pmformalizer{chaunguyen}{}
\pmtitle{Stern prime}
\pmrecord{4}{38446}
\pmprivacy{1}
\pmauthor{PrimeFan}{13766}
\pmtype{Definition}
\pmcomment{trigger rebuild}
\pmclassification{msc}{11N05}
\pmdefines{Stern prime}

\endmetadata

\usepackage{amssymb}
\usepackage{amsmath}
\usepackage{amsfonts}
\usepackage{cnl}
\usepackage{xcolor}
%\usepackage{graphicx}
%%%%\usepackage{xypic}

% TITLE AUTHOR DATE
\title{Number Theory,\\ Stern pprime}
\date{October 24, 2019}
\author{Chau Nguyen (chaunguyen)}

%- DOCUMENT

\begin{document}
\parskip =\baselineskip
	
% Local Defs for Coprime definition 
\def\natdiv#1#2{{#1}\mathrel{|}{#2}}
\def\intpow#1#2{{#1}^{#2}}
\parindent=0pt
	
\begin{cnl}
		
\Cnlinput{../Tex2CNL/package/cnlinit}
		
\bigskip
		
%-% BEGIN

\lsection{Stern prime }
		
In this section, let $b,\ d,\ m,\ p$ be integers.

\dfn{
We say that $d$ is \df{positive } iff $d > 0$.
}

\dfn{
We say that $b$ is \df{nonzero} iff $b \ne 0$.
}
		
\dfn{
We say that $d$ \df{divides} $m$ iff $d\ne 0$ and there exists an integer $r$ such that $m= d\*r$.
} 
	
\dfn{
We write  $\natdiv{d}{m}$ iff $d$ divides $m$.	
}

\dfn{
We say  $d$ is a \df{divisor} of $m$ iff $d$ divides $m$. 
}

\dfn{
Assume that $p$ is greater than $1$.  Then we say that $p$
is a \df{prime} iff each positive divisor of $p$ is equal to $1$ or equal to $p$.
}

\dfn{
Assume that $p$ is a prime.  Then we say that $p$
is a \df{Stern prime} iff there exist no prime $q$ and nonzero integer $b$ such that $p  = 2\*\intpow{b}{2} + q$.
}

\begin{remark}
These primes were first studied by Moritz Abraham Stern, in connection to a lesser known conjecture of Goldbach's. Like other mathematicians of the time, Stern considered 1 to be a prime number. Thus his list of Stern primes read thus: 2, 17, 137, 227, 977, 1187, 1493. A century later the list has been amended to include 3 (as in A042978 of Sloane's OEIS) but no terms larger than 1493 have been found. The larger of a twin prime is not a Stern prime.
\end{remark}

\end{cnl}




\end{document}
