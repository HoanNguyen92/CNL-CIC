\documentclass[12pt]{article}
\usepackage{pmmeta}
\pmcanonicalname{ChineseHypothesis}
\pmcreated{2013-03-22 18:11:08}
\pmmodified{2019-11-24}
\pmowner{FourDozens}{21006}
\pmmodifier{FourDozens}{21006}
\pmformalizer{chaunguyen}{}
\pmtitle{Chinese hypothesis}
\pmrecord{6}{40757}
\pmprivacy{1}
\pmauthor{FourDozens}{21006}
\pmtype{Definition}
\pmcomment{trigger rebuild}
\pmclassification{msc}{11A51}

\endmetadata




\usepackage{amssymb}
\usepackage{amsmath}
\usepackage{amsfonts}
\usepackage{cnl}
\usepackage{xcolor}
%\usepackage{graphicx}
%%%%\usepackage{xypic}


\newtheorem{hypothesis}{Hypothesis}
%- DOCUMENT

\begin{document}
\parskip =\baselineskip
	
% Local Defs 

\def\natdiv#1#2{{#1}\mathrel{|}{#2}}
	
\begin{cnl}
		
\Cnlinput{../Tex2CNL/package/cnlinit}
		
\bigskip
		
%-% BEGIN

\lsection{ChineseHypothesis}
		
In this section, let $d,\ m$ be integers.
		
\dfn{
We say that $d$ \df{divides} $m$ iff $d\ne 0$ and there exists an integer $r$ such that $m= d\*r$.
}
	
\dfn{
We write  $\natdiv{d}{m}$ iff $d$ divides $m$.	
}

\dfn {
We say  $d$ is a \df{divisor} of $m$ iff $d$ divides $m$. 
}

\dfn{
Assume that $p$ is an integer greater than $1$.  Then we say that $p$
is a \df{prime} iff each divisor of $p$ is either equal to $1$ or equal to $p$.
}

\deflabel{intpownat}
Assume that $b$ is an integer and $n$ is a natural number.  Then we define
$\intpownat{b}{n}\assign$
\par$\match\ n$ with
\begin{envMatch}
	\firstmatchitem $0$ &$\assign$& $1$
	\matchitem $m + 1$ &$\assign$& $b \* \intpownat{b}{m}$\texstop
\end{envMatch}\cnlstop
This exists by recursion.
\end{definition}

\begin{hypothesis}[ChineseHypothesis]
We say an integer $n$ is a prime iff $\natdiv {n}{\intpownat{2}{n}-2}. 
\end{hypothesis}

\begin{remark}
 By Fermat's little theorem we have that $2^p \equiv 2 \mod p$, so that means $n$ does divide $2^n- 2$ if $n$ is prime. However, if $n$ is composite Fermat's little theorem does not rule out that $n$ could divide $2^n-2$. The Chinese hypothesis checks out for the small powers of two. The first counterexample is $n=341$, but since $2^{341}$ has more than a hundred digits, it wasn't easy to check it back in the 18th century when this test was first proposed. Though back then they attributed it to ancient Chinese mathematicians, hence the name.
\end{remark}

\end{cnl}



\end{document}