\documentclass[12pt]{article}
\usepackage{pmmeta}
\pmcanonicalname{FermatsLittleTheorem}
\pmcreated{2013-03-22 11:45:08}
\pmmodified{2019-12-18}
\pmowner{CWoo}{3771}
\pmmodifier{CWoo}{3771}
\pmformalizer{HoanNguyen}{} 
\pmtitle{Fermat's little theorem}
\pmrecord{13}{30195}
\pmprivacy{1}
\pmauthor{CWoo}{3771}
\pmtype{Theorem}
\pmcomment{trigger rebuild}
\pmclassification{msc}{11-00}
\pmclassification{msc}{16E30}
\pmsynonym{Fermat's theorem}{FermatsLittleTheorem}
%\pmkeywords{number theory}
\pmrelated{EulerFermatTheorem}
\pmrelated{ProofOfEulerFermatTheoremUsingLagrangesTheorem}
\pmrelated{FermatsTheoremProof}
\pmrelated{PolynomialCongruence}

\endmetadata

\usepackage{amssymb}
\usepackage{amsmath}
\usepackage{amsfonts}
\usepackage{graphicx}
%%%%\usepackage{xypic}

\usepackage{amsthm}


\usepackage{cnl}
\usepackage{xcolor}

% there are many more packages, add them here as you need them

% define commands here
\newcommand{\intpow}[2]{{#1}^{#2}}

\begin{document}
	
\parskip=\baselineskip

\begin{cnl}
\Cnlinput{../TeX2CNL/package/cnlinit}
		
\bigskip
		
In this section, let $a$, $p$ be integers.		
		
		
%[Fermat's little theorem] 
%[Fermat's little theorem] 
\thm{If $p$ is a prime and $p \nmid a$, then $$\intpownat{a}{p-1} \equiv 1 \pmod {p}.$$}

\thm{If $p$ is a prime, then $$\intpow{a}{p} \equiv a \pmod{p}$$}

%\thm{If $p$ is a prime, then $$\intpow{a}{p} \equiv a \pmod{p}$$. Moreover, if $p \nmid a$, then  $$\intpownat{a}{p-1} \equiv 1 \pmod {p}.$$}

%If we take away the condition that $p\nmid a$, then we have the congruence relation $$a^p\equiv a \pmod{p}$$ instead.  

	\begin{remark}
	\begin{itemize}
		\item Although Fermat first noticed this property, he never actually proved it.  There are several different ways to directly prove this theorem, but it is really just a corollary of the Euler theorem.
		\item More generally, this is a statement about finite fields: if $K$ is a finite field of order $q$, then $\inpow{a}{p-1} = 1$ for all $0\ne a\in K$.  More succinctly, the group of units in a finite field is cyclic.  If $q$ is prime, then we have Fermat's little theorem.
		\item While it is true that $p$ prime implies the congruence relation above, the converse is false (as hypothesized by ancient Chinese mathematicians).  A well-known example of this is provided by setting $a=2$ and $p=341=11\times 31$.  It is easy to verify that $\inpow{2}{341}\equiv 2 \pmod{341}$.  A positive integer $p$ satisfying $\inpow{a}{p-1}  \equiv 1 \pmod{p}$ is known as a pseudoprime of base $a$.  Fermat little theorem says that every prime is a pseudoprime of any base not divisible by the prime.
	\end{itemize}
\end{remark}

%\begin{thebibliography}{7}
%\bibitem{hs} H. Stark, {\em An Introduction to Number Theory}. The MIT Press (1978)
%\end{thebibliography}




\end{cnl}
%%%%%
%%%%%
\end{document}