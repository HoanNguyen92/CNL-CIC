

\documentclass[12pt]{article}
\usepackage{pmmeta}
\pmcanonicalname{Coprime}
\pmcreated{2013-03-22 11:46:13}
\pmmodified{2019-11-22}
\pmowner{drini}{3}
\pmmodifier{drini}{3}
\pmformalizer{chaunguyen}{}
\pmtitle{coprime}
\pmrecord{9}{30225}
\pmprivacy{1}
\pmauthor{drini}{3}
\pmtype{Definition}
\pmcomment{trigger rebuild}
\pmclassification{msc}{11-00}
\pmclassification{msc}{53B20}
\pmclassification{msc}{53C25}
\pmclassification{msc}{53C17}
\pmclassification{msc}{53B21}
\pmsynonym{relatively prime}{Coprime}
\pmdefines{divides}
\pmdefines{divisor}
\pmdefines{common divisor}
\pmdefines{greatest common divisor}
\pmdefines{coprime}
\pmdefines{relatively prime}
%\pmkeywords{Integers}
%\pmkeywords{Divisibility}

\endmetadata

\usepackage{amssymb}
\usepackage{amsmath}
\usepackage{amsfonts}
\usepackage{cnl}
\usepackage{xcolor}
%\usepackage{graphicx}
%%%%\usepackage{xypic}

% TITLE AUTHOR DATE
\title{Number Theory,\\ Coprime}
\date{October 22, 2019}
\author{Chau Nguyen (chaunguyen)}

%- DOCUMENT

\begin{document}
\parskip =\baselineskip

% Local Defs for Coprime definition 
\def\natdiv#1#2{{#1}\mathrel{|}{#2}}

\begin{cnl}

\Cnlinput{../Tex2CNL/package/cnlinit}

\bigskip

%-% BEGIN
\lsection{Coprime}

In this section, let $d,\ m,\ p,\ q$ be integers.

%-% Definition

\deflabel{divides}
Assume that $d,\ m$ are integer numbers. Then we say that $d$ \df{divides} $m$ iff $d\ne 0$ and there exists $r$
such that $m= d\*r$.
\end{definition}

\begin{remark}				
We write  $\natdiv{d}{m}$ iff $d$ divides $m$.				
We say  $d$ is a \df{divisor of} $m$ iff $d$ divides $m$
\end{remark}

\deflabel{common divisor}
Assume $d,\ p,\ q$ are integers. Then we say that $d$ is \df{a common divisor} of $p$ and $q$  iff $d$ divides $p$  and $d$ divides $q$. 
\end{definition}

\begin{remark}
We now introduce the greatest common divisor: 
\end{remark}

\deflabel{greatest common divisor}
Assume $d,\ p,\ q$ are integers. Then we say that $d$ is \df{a greatest common divisor} of $p$ and $q$  
iff $d$ is a common divisor of $p$ and $q$ and 
there exists no $r$ such that $r$ is a common divisor of $p$ and $q$ and $r >d$. This greatest common divisor exists and is unique. 
\end{definition}

\begin{remark}
We write $gcd (p,\q) = d$ iff $d$ is the greatest common divisor of $p$ and $q$.
\end{remark}

\deflabel{coprime}
Assume $p,\ q$ are integers. Then we say that $p,\ q$ are  \df{coprime} iff  $gcd(p,\ q)=1$. 
\end{definition}

\begin{remark}
It is also said that $p,\ q$ are \df{relatively prime}.
\end{remark}

\end{cnl}




\end{document}