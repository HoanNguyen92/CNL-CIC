\documentclass[12pt]{article}
\usepackage{pmmeta}
\pmcanonicalname{FactorialPrime}
\pmcreated{2013-03-22 16:19:19}
\pmmodified{2019-11-24}
\pmowner{PrimeFan}{13766}
\pmmodifier{PrimeFan}{13766}
\pmformalizer{chaunguyen}{}
\pmtitle{factorial prime}
\pmrecord{6}{38449}
\pmprivacy{1}
\pmauthor{PrimeFan}{13766}
\pmtype{Definition}
\pmcomment{trigger rebuild}
\pmclassification{msc}{11A41}
\pmclassification{msc}{05A10}
\pmclassification{msc}{11B65}
\pmdefines{factorial prime}
\endmetadata

\usepackage{amssymb}
\usepackage{amsmath}
\usepackage{amsfonts}
\usepackage{cnl}
\usepackage{xcolor}
%\usepackage{graphicx}
%%%%\usepackage{xypic}

%- DOCUMENT

\begin{document}
\parskip =\baselineskip
	
% Local Defs 
\def\natdiv#1#2{{#1}\mathrel{|}{#2}}
\def\factorial#1{{#1!}}
\parindent=0pt
	
\begin{cnl}
		
\Cnlinput{../Tex2CNL/package/cnlinit}
		
\bigskip
		
%-% BEGIN

\lsection{FactorialPrime}
		
In this section, let $d,\ m,\ p$ be integers.
		
\dfn{ We say that $d$ \df{divides} $m$ iff $d\ne 0$ and there exists
  an integer $r$ such that $m= d\*r$.  }

\dfn{
We say that $d$ is \df{positive} iff $d > 0$.
}
	
\dfn{
We write  $\natdiv{d}{m}$ iff $d$ divides $m$.	
}

\dfn{
We say  $d$ is a \df{divisor} of $m$ iff $d$ divides $m$. 
}

\dfn{ Assume that $p$ is an integer greater than $1$.  Then we say
  that $p$ is a \df{prime} iff each positive divisor of $p$ is either
  equal to $1$ or equal to $p$.  }

\dfn{
Assume that $n$ is a natural number.  Then we define
$\factorial{n}\assign$
\par$\match\ n$ with
\begin{envMatch}
\firstmatchitem $0$ &$\assign$& $1$
\matchitem $m + 1$ &$\assign$& $(m+1) \* \factorial{m}$\texstop
\end{envMatch}\cnlstop
This exists by recursion.
}

\dfn{ Assume that $p$ is a prime.  Then we say that $p$ is a
  \df{factorial prime} iff there exists a natural number $n$ such that
  $p$ is equal to $\factorial {n}+ 1 $ or $p$ equal to $\factorial {n}
  - 1 $.  }

\begin{remark}
The first few factorial primes are: 2, 3, 5, 7, 23, 719, 5039,
39916801, 479001599, 87178291199 (sequence A088054 in the OEIS). It is
conjectured that only for $n = 3$ are both $n! - 1$ and $n! + 1$ both
primes.

Factorial primes have a r\^ole in an argument that 1 is not a prime
number. If $n$ is a positive integer and $p$ is a prime number, $n! +
p$ is never a prime for $p < n$, because obviously it will be a
multiple of $p$, just as $n!$ is. But $n! + 1$, even though it
certainly is a multiple of 1, can be a prime, specifically, a
factorial prime. (The same is also true if we subtract instead of
add).
\end{remark}

\end{cnl}




\end{document}
