%% Style guidelines
% Each lemma starts with optional name [], then \label, then optional
% \preamble{}, then statement.

% two blank lines before a \begin starting a new paragraph.  Otherwise one.
% two blank lines before new (sub)section

%% VERBATIM
% fancy verbatim environment is required.

%\DefineShortVerb[formatcom=\color{red},fontfamily=mathtt]{\!}

%\DefineShortVerb[formatcom=\color{blue},fontfamily=mathtt]{\|}

%\newenvironment{evMatch}%
%               {\begin{itemize}\renewcommand\labelitemi{\normalfont|}}%
%               {\end{itemize}}

\newenvironment{envMatch}%
               {\par\begin{tabular}{@{\quad\normalfont| }lll}}%
               {\end{tabular}\par}



%\renewcommand\labelitemi{\normalfont|}}


%% ELLIPSIS, 
% based on Luis Berlioz's query
% https://tex.stackexchange.com/questions/503731/how-to-define-a-macro-that-takes-the-definition-of-a-macro-as-an-argument
% \usepackage{amsmath}
% \usepackage{xparse}

\ExplSyntaxOn
\NewDocumentCommand{\ellipsis}{mmmm}
 {% #1 = main term
  % #2 = first index
  % #3 = last index
  % #4 = operation
  \group_begin:
  \lucas_ellipsis:nnnn { #1 } { #2 } { #3 } { #4 }
  \group_end:
 }
\cs_new:Nn \lucas_ellipsis:nnnn
 {
  \cs_set:Nn \__lucas_ellipsis_term:n { #1 }
  \__lucas_ellipsis_term:n { #2 }
  #4 \dots #4
  \__lucas_ellipsis_term:n { #3 }
 }
\ExplSyntaxOff

%\begin{document}

%$\ellipsis{x^{#1}}{0}{5}{+}$

%$\ellipsis{x_{#1}}{0}{5}{+}$

%$\ellipsis{(x_{#1}+y_{#1}i)}{1}{n}{}$

%\end{document}
%



%% CNL CONTROL SEQUENCES

\def\ignoreOptionAndCS[#1]#2{}

\def\onearg#1{(onearg:#1)} % for debugging.

\ignoreOptionAndCS[3]\onearg


\def\ignoreOne#1{}

%\ignoreOne\onearg C

\def\ignoreOptionOrCS{%
\futurelet\nextToken\chooseBranch}


\def\chooseBranch{%
\let\next=\relax
\ifx\nextToken [%
 \let\next=\ignoreOptionAndCS%
\else%
 \let\next=\ignoreOne%
\fi%
\next%
}

\def\ignoreOptionAndCSS[#1]#2#3{}
\def\ignoreTwo#1#2{}
\def\ignoreOptionOrCSS{%
\futurelet\nextToken\chooseBranchTwo%
}
\def\chooseBranchTwo{%
\let\next=\relax
\ifx\nextToken [%
 \let\next=\ignoreOptionAndCSS%
\else%
 \let\next=\ignoreTwo%
\fi%
\next%
}

\let\CnlExpand=\ignoreOptionOrCS
\let\CnlNoExpand=\ignoreOptionOrCS
\let\CnlDelete=\ignoreOptionOrCS
\let\CnlCustom=\ignoreOptionOrCSS
\let\CnlDef=\ignoreOptionOrCSS
\let\CnlError=\ignoreOptionOrCS
\def\CnlEnvirDelete#1{}


%% ENVIRONMENTS

\newenvironment{cnl}{}{}
% \leavevmode\par is to make remark work 
% when it is the first item in a subsection.

\newenvironment{remark}%{}{} %temp debug
{\leavevmode\par\begin{tabular}{|p{13cm}}\parskip=\baselineskip{\bf Remark.}}
{\end{tabular}}



\newcommand{\var}[1]{#1}
\newcommand{\id}[1]{#1}
\newcommand{\prefix}[1]{}
\newcommand{\app}[1]{#1}
\newcommand{\CnlList}[1]{#1} % was list
\newcommand{\parenI}[1]{#1}
\renewcommand{\*}{\,}

\renewcommand{\iff}{\leftrightarrow}
\newcommand{\Prop}{\text{\tt Prop}}
\newcommand{\Type}{\text{\tt Type}}
\newcommand{\fld}{.}
\newcommand{\dequiv}{\mathrel{:=}} %{\mathrel{:\equiv}}
\newcommand{\assign}{\mathrel{:=}} %{\mathrel{:\equiv}}
\newcommand{\Nat}{\ensuremath{{\mathbb N}}}
\newcommand{\Real}{\ensuremath{{\mathbb R}}}
\newcommand{\Bool}{\ensuremath{{\mathbb {B}}}}
\newcommand{\df}[1]{\text{\bf #1}}
\newcommand{\h}[1]{\text{\color{red} #1}}
\newcommand{\join}{\lor}
\newcommand{\Mid}{\mathrel{\|}}
\newcommand{\comment}[1]{\%- \nobreak{#1}}
%\renewcommand{\~}{\ }
\newcommand{\ignore}[1]{}
\newcommand{\remark}[1]{(#1)}
%\renewcommand{\_}{\textunderscore}
\renewcommand\labelitemi{--}
\renewcommand{\qed}{\ensuremath{\square}}

\def\deflabel#1{\begin{definition}[#1]\label{#1}}
\def\thmlabel#1{\begin{theorem}[#1]\label{#1}}
\def\namelabel#1{[#1]\label{#1}}
\newcommand{\keyword}[1]{{\text{\bf{#1}}}}
\newcommand{\fun}{\keyword{fun}}
\newcommand{\match}{\keyword{match}}
\newcommand{\ifcond}{\keyword{if}}
\newcommand{\thencond}{\keyword{then}}
\newcommand{\elsecond}{\keyword{else}}
\newcommand{\otherwise}{\keyword{otherwise}}
\newcommand{\function}{\keyword{function}}
\newcommand{\quot}{\keyword{quot}}
\newcommand{\where}{\keyword{where}}
\newcommand{\blank}{\h{\_}}

% Accents- Use these versions to fuse with variable name in CNL image.
% \mathcheck{c} --> V__c_mathcheck
% \check{c} --> \check{c} (function application)
\newcommand{\mathhat}{\hat}
\newcommand{\mathwidehat}{\widehat}
\newcommand{\mathcheck}{\check}
\newcommand{\mathtilde}{\tilde}
\newcommand{\mathwidetilde}{\widetilde}
\newcommand{\mathacute}{\acute}
\newcommand{\mathgrave}{\grave}
\newcommand{\mathdot}{\dot}
\newcommand{\mathddot}{\ddot}
\newcommand{\mathbreve}{\breve}
\newcommand{\mathbar}{\bar}
\newcommand{\mathvec}{\vec}



\begin{cnl}


%% We start with some that will eventually be moved to a standard library.

\CnlCustom[1]\parenI{ (#1) }
\CnlCustom\*{*}


% Prohibited control sequences. 
% That is, they should remain outside the cnl environment.
% We should list many TeX primitive control sequences here.

\CnlError\if
\CnlError\else
\CnlError\fi
\CnlError\let
\CnlError\futurelet
\CnlError\expandafter
\CnlError\afterassignment
\CnlError\usepackage
% etc. 

%
\CnlDelete[1]\phantom
\CnlNoExpand[1]\section

% math modes and space
\CnlDelete\ensuremath
\CnlDelete\text
\CnlDelete\thinmuskip
\CnlDelete\medmuskip
\CnlDelete\thickmuskip
\CnlDelete\quad
\CnlDelete\qquad
\CnlDelete\,
\CnlDelete\:
\CnlDelete\;
\CnlDelete\!
\CnlDelete\ %space
\CnlDelete\enspace
\CnlDelete[1]\hspace
\CnlDelete\hfil
\CnlDelete\hfill
\CnlDelete\thinspace
\CnlDelete\left
\CnlDelete\right
\CnlDelete\big
\CnlDelete\Big
\CnlDelete\bigg
\CnlDelete\Bigg
\CnlDelete\allowdisplaybreaks


%other  spacing
\CnlDelete\noindent
\CnlDelete\indent
\CnlDelete[1]\vspace
\CnlDelete\null
\CnlDelete\break
\CnlDelete\newline
\CnlDelete\newpage
\CnlDelete\vfil
\CnlDelete\vfill
\CnlDelete\smallskip
\CnlDelete\medskip
\CnlDelete\bigskip
\CnlDelete[2]\rule

\CnlEnvirDelete{remark}
\CnlEnvirDelete{summary}
\CnlEnvirDelete{tikzpicture}
\CnlEnvirDelete{fancyvrb}

\end{cnl}






