
\documentclass[12pt]{article}


%- PACKAGES
\usepackage{url}
\usepackage{amsmath}
\usepackage{amsthm}
\usepackage{amssymb}
\usepackage{xcolor}
\usepackage{currfile}
\usepackage{fancyvrb}
\usepackage{xparse} % needed for \ellipsis control sequence in cnl-style
\usepackage{enumitem} % for topsep=0pt
\usepackage{colonequals}

% for underscores https://texfaq.org/FAQ-underscore
\usepackage{lmodern}
\usepackage[T1]{fontenc}
\usepackage{textcomp}
\usepackage{lineno}

\usepackage[
bookmarksopen,
bookmarksdepth=2,
%breaklinks=true
colorlinks=true,
urlcolor=blue]{hyperref}

% GLOBAL FORMATTING
%\linenumbers
\parindent=0pt
\parskip=0.5\baselineskip
\raggedbottom

% TITLE AUTHOR DATE
\title{Sample Controlled Natural Language text,\\ including real numbers and Sylows theorems}

\date{September 14, 2019}
\author{Thomas Hales}

%- THEOREMS
\newtheorem{definition}{Definition}
\newtheorem{theorem}[definition]{Theorem}
\newtheorem{lemma}[definition]{Lemma}
\numberwithin{definition}{section}
%\newtheorem{specification}[definition]{Specification}


%- DOCUMENT

\begin{document}
\maketitle

\setcounter{tocdepth}{1}
\tableofcontents
\newpage


%\renewcommand\labelitemi{\normalfont|}}


%% ELLIPSIS, 
% based on Luis Berlioz's query
% https://tex.stackexchange.com/questions/503731/how-to-define-a-macro-that-takes-the-definition-of-a-macro-as-an-argument
% \usepackage{amsmath}
% \usepackage{xparse}

\ExplSyntaxOn
\NewDocumentCommand{\ellipsis}{mmmm}
 {% #1 = main term
  % #2 = first index
  % #3 = last index
  % #4 = operation
  \group_begin:
  \lucas_ellipsis:nnnn { #1 } { #2 } { #3 } { #4 }
  \group_end:
 }
\cs_new:Nn \lucas_ellipsis:nnnn
 {
  \cs_set:Nn \__lucas_ellipsis_term:n { #1 }
  \__lucas_ellipsis_term:n { #2 }
  #4 \dots #4
  \__lucas_ellipsis_term:n { #3 }
 }
\ExplSyntaxOff

%\begin{document}

%$\ellipsis{x^{#1}}{0}{5}{+}$

%$\ellipsis{x_{#1}}{0}{5}{+}$

%$\ellipsis{(x_{#1}+y_{#1}i)}{1}{n}{}$

%\end{document}
%

%% SUSPEND itemize
% https://tex.stackexchange.com/questions/135726/intertext-like-command-in-enumerate-environment


%% CNL CONTROL SEQUENCES

\def\ignoreOptionAndCS[#1]#2{}

\def\onearg#1{(onearg:#1)} % for debugging.

\ignoreOptionAndCS[3]\onearg


\def\ignoreOne#1{}

%\ignoreOne\onearg C

\def\ignoreOptionOrCS{%
\futurelet\nextToken\chooseBranch}


\def\chooseBranch{%
\let\next=\relax
\ifx\nextToken [%
 \let\next=\ignoreOptionAndCS%
\else%
 \let\next=\ignoreOne%
\fi%
\next%
}
\ignoreOne{]}

\def\ignoreOptionAndCSS[#1]#2#3{}
\def\ignoreTwo#1#2{}
\def\ignoreOptionOrCSS{%
\futurelet\nextToken\chooseBranchTwo%
}
\def\chooseBranchTwo{%
\let\next=\relax
\ifx\nextToken [%
 \let\next=\ignoreOptionAndCSS%
\else%
 \let\next=\ignoreTwo%
\fi%
\next%
}

\let\CnlExpand=\ignoreOptionOrCS
\let\CnlNoExpand=\ignoreOptionOrCS
\let\CnlDelete=\ignoreOptionOrCS
\let\CnlCustom=\ignoreOptionOrCSS
\let\CnlDef=\ignoreOptionOrCSS
\let\CnlError=\ignoreOptionOrCS
\def\CnlEnvirDelete#1{}


%% ENVIRONMENTS

\newenvironment{cnl}{\centerline{\bf CNL text starts here.}}
{\centerline{\bf CNL text ends here.}}
% \leavevmode\par is to make remark work 
% when it is the first item in a subsection.

\newenvironment{remark}%{}{} %temp debug
{\leavevmode\par\begin{tabular}{|p{13cm}}\parskip=\baselineskip{\bf Remark.}}
{\end{tabular}}

\newenvironment{envMatch}%
               {\par\begin{tabular}{@{\quad\normalfont| }lll}}%
               {\end{tabular}\par}

\newenvironment{structure}%
{\begin{itemize}[topsep=0pt]}
{\end{itemize}}

\newcommand{\interitem}[1]{\end{itemize}#1\begin{itemize}[topsep=0pt]}


\newenvironment{make}%                             
{\begin{itemize}[topsep=0pt]}
{\end{itemize}}

%\newenvironment{format}%
%{\begin{align*}}{\end{align*}}

% ENVIRONMENT ITEMS
\renewcommand\labelitemi{--}
\newcommand{\matchitem}{\\}
\newcommand{\firstmatchitem}{}
\newcommand{\firstitem}{\item}

% LABELS
\def\lsection#1{\section{#1}\label{#1}}
\def\lsubsection#1{\subsection{#1}\label{#1}}
\def\lsubsubsection#1{\subsubsection{#1}\label{#1}}
\def\deflabel#1{\begin{definition}[#1]\label{#1}}
\def\thmlabel#1{\begin{theorem}[#1]\label{#1}}
\def\namelabel#1{[#1]\label{#1}}

% SPECIALS
\newcommand{\var}[1]{#1}
\newcommand{\id}[1]{#1}
\newcommand{\prefix}[1]{}
\newcommand{\app}[1]{#1}
\newcommand{\CnlList}[1]{#1} % was list
\newcommand{\parenI}[1]{#1}
\newcommand{\df}[1]{\text{\bf #1}}
\newcommand{\h}[1]{\text{\color{red} #1}}
\newcommand{\Mid}{\mathrel{\|}}
\newcommand{\ignore}[1]{}
\newcommand{\parenthetical}[1]{(#1)}
\newcommand{\funmapsto}[2]{#1\mapsto #2}
\newcommand{\funalign}{}
\newcommand{\setcomp}[2]{\{#1 \mid #2\}}
\newcommand{\setenum}[1]{\{#1\}}
\newcommand{\wherearg}[1]{\where\ $\{#1\}$}
\newcommand{\optarg}[1]{\{#1\}}
\newcommand{\texstop}{.}
\newcommand{\texcomma}{,}
\newcommand{\cnlstop}{}
\newcommand{\plural}{/-}
\newcommand\caseif[2]{#1 & \ifcond\ #2}
\newcommand\caseotherwise[1]{#1&\otherwise}
\newcommand\onlyTeX[1]{#1}

% SYMBOLS
\newcommand{\join}{\lor}
\newcommand{\blank}{\h{\_}}
\renewcommand{\qed}{\ensuremath{\square}}
\newcommand{\inv}[1]{{#1}^{-1}}
\newcommand{\nullbrack}{[]}
\renewcommand{\*}{\,}
\renewcommand{\iff}{\leftrightarrow}
\newcommand{\Prop}{\text{\tt Prop}}
\newcommand{\Type}{\text{\tt Type}}
\newcommand{\Bool}{\text{\tt Bool}}
\newcommand{\assign}{\colonequals} % {colonequals}
\newcommand{\cons}{\mathrel{\coloncolon}} % {colonequals}
\newcommand{\Nat}{\ensuremath{{\mathbb N}}}
\newcommand{\Real}{\ensuremath{{\mathbb R}}}

% KEYWORDS
\newcommand{\keyword}[1]{{\text{\bf{#1}}}}
\newcommand{\Make}{\keyword{make}}
\newcommand{\fun}{\keyword{fun}}
\newcommand{\match}{\keyword{match}}
\newcommand{\ifcond}{\keyword{if}}
\newcommand{\thencond}{\keyword{then}}
\newcommand{\elsecond}{\keyword{else}}
\newcommand{\otherwise}{\keyword{otherwise}}
\newcommand{\function}{\keyword{function}}
\newcommand{\quot}{\keyword{quot}}
\newcommand{\etc}{\keyword{etc}}
\newcommand{\where}{\keyword{where}}

% ACCENTS
% Use these versions to fuse with variable name in CNL image.
% \mathcheck{c} --> V__c_mathcheck
% \check{c} --> \check{c} (function application)
\newcommand{\mathhat}{\hat}
\newcommand{\mathwidehat}{\widehat}
\newcommand{\mathcheck}{\check}
\newcommand{\mathtilde}{\tilde}
\newcommand{\mathwidetilde}{\widetilde}
\newcommand{\mathacute}{\acute}
\newcommand{\mathgrave}{\grave}
\newcommand{\mathdot}{\dot}
\newcommand{\mathddot}{\ddot}
\newcommand{\mathbreve}{\breve}
\newcommand{\mathbar}{\bar}
\newcommand{\mathvec}{\vec}

%\renewcommand{\_}{\textunderscore}
%\newcommand{\comment}[1]{\%- \nobreak{#1}}
%\renewcommand{\~}{\ }
%\newcommand{\Bool}{\ensuremath{{\mathbb {B}}}}
%\newcommand{\cons}{\mathrel{{\bf :\hskip-1.1pt:}}}
%\newcommand{\fld}{.} %use \h{. }
%\newcommand{\assign}{\mathrel{:=}} %{\mathrel{:\equiv}}
%\newcommand{\caseif}{{\bf if}}
%\newcommand{\caseotherwise{{\bf otherwise}}

% Local Defs for Sylows Theorems
\def\natdiv#1#2{{#1}\mathrel{|}{#2}}
\def\natpow#1#2{{#1}^{#2}}
\def\conj#1#2{{#1}{#2}{#1}^{-1}}
\def\mult#1#2{m}
\def\Syl#1#2{\operatorname{Syl}_{#1}(#2)}
\def\SylN#1#2{n_{#1}(#2)}
\def\Nz#1#2{|N(#1,#2)|}
\def\realabs#1{|#1|}
\def\grouporder#1{|#1|}


\begin{cnl}


%% We start with some that will eventually be moved to a standard library.

\CnlCustom[1]\parenI{ (#1) }
\CnlCustom\*{*}
\CnlCustom[1]\lsection{Section \concat{}{#1} .}
\CnlCustom[1]\lsubsection{Subsection \concat{}{#1} .}
\CnlCustom[1]\lsubsubsection{Subsubsection \concat{}{#1} .}
\CnlCustom[1]\deflabel{\begin{definition}\label{#1}}
\CnlCustom[1]\df{ #1 }
\CnlCustom[1]\h{ #1 }
\CnlCustom\where{where}
\CnlCustom\ifcond{if}
\CnlCustom\otherwise{true}
\CnlCustom\elsecond{else}
\CnlCustom\where{where}
\CnlCustom\assign{ \concat{}{:=} }
\CnlCustom\cons{\concat{}{::}} 
\CnlCustom\plural{\concat{}{/-}}
\CnlCustom\wherearg{where}
\CnlCustom\fun{fun}
\CnlCustom[2]\funmapsto{(fun #1 \assign #2)}
%\funalign{A}&{\to}{B}\\{a}{\mapsto}{b}
\CnlCustom[6]\funalign{(fun (#4 : #1) \assign (#6 : #3))}
\CnlCustom\Prop{Prop}
\CnlCustom\Type{Type}
\CnlCustom\Bool{Bool}
\CnlCustom\iff{iff}
\CnlCustom\optarg{}
\CnlCustom\match{match}



\CnlCustom\function{function}
\CnlCustom\Make{make}
\CnlCustom\etc{\_}
\CnlCustom\Nat{Nat}
\CnlCustom\Real{Real}
\CnlCustom\quot{quot}
\CnlCustom\alpha{V--alpha}
\CnlCustom\beta{V--beta}
\CnlCustom\prefix{\concat}
\CnlCustom\cnlstop{.}
%\CnlCustom\nullbrack{\concat{}{[]}}
\CnlCustom\matchitem{\alt}
\CnlCustom\firstmatchitem{\alt}
\CnlCustom[2]\caseif{#2 \assign #1}
\CnlCustom[1]\caseotherwise{true \assign #1}
\CnlCustom\rightarrow{\imply}
%\CnlNoExpand[1]\section


% Prohibited control sequences. 
% That is, they should remain outside the cnl environment.
% We should list many TeX primitive control sequences here.

% CNLERROR
\CnlError\if
\CnlError\else
\CnlError\fi
\CnlError\let
\CnlError\futurelet
\CnlError\afterassignment
\CnlError\usepackage
% etc. 

% CNLDELETE
\CnlDelete\expandafter
\CnlDelete[1]\interitem
\CnlDelete\texstop
\CnlDelete[1]\onlyTeX
\CnlDelete[1]\phantom
\CnlDelete\firstitem
\CnlDelete\texstop
\CnlDelete\texcomma
% math modes and space
\CnlDelete\ensuremath
\CnlDelete\text
\CnlDelete\thinmuskip
\CnlDelete\medmuskip
\CnlDelete\thickmuskip
\CnlDelete\quad
\CnlDelete\qquad
\CnlDelete\,
\CnlDelete\:
\CnlDelete\;
\CnlDelete\!
\CnlDelete\ %space
\CnlDelete\enspace
\CnlDelete[1]\hspace
\CnlDelete\hfil
\CnlDelete\hfill
\CnlDelete\thinspace
\CnlDelete\left
\CnlDelete\right
\CnlDelete\big
\CnlDelete\Big
\CnlDelete\bigg
\CnlDelete\Bigg
\CnlDelete\allowdisplaybreaks
%other  spacing
\CnlDelete\noindent
\CnlDelete\indent
\CnlDelete[1]\vspace
\CnlDelete\null
\CnlDelete\break
\CnlDelete\newline
\CnlDelete\newpage
\CnlDelete\vfil
\CnlDelete\vfill
\CnlDelete\smallskip
\CnlDelete\medskip
\CnlDelete\bigskip
\CnlDelete[2]\rule
\CnlDelete[1]\parenthetical

%\CnlDelete\par

\CnlEnvirDelete{remark}
\CnlEnvirDelete{summary}
\CnlEnvirDelete{tikzpicture}
\CnlEnvirDelete{fancyvrb}




%-% BEGIN

\begin{remark}
This is a preliminary text.  It
writes some mathematical statements according to a
controlled natural language for mathematics.
\end{remark}


\lsection{Generalities}

\begin{remark}
This article is an experiment in writing mathematics in a computer
readable format.  It is a preliminary step in the creation of a
controlled natural language based on English with semantics based on
the Calculus of Inductive Constructions (CiC).  We have in mind the
dialect of CiC described in Carneiro's CMU master's thesis, and
implemented in the Lean theorem prover.

The controlled natural language is inspired by earlier CNLs such as
Mizar, Evidence Algorithm (EA), Forthel (formula theory language), SAD
(system of automated deduction), Naproche (natural language proof
checking), etc.

The mathematical content here is not important.  What matters is the
language and the stages of transformation from English to CiC.

Text placed in remarks (such as this text) are not part of the
controlled natural language and are ignored by the translation engine.
\end{remark}




\begin{remark}
Structures (inductive types with a single constructor) are the basic
building block for all mathematical structures. Whenever a
mathematician defines a mathematical structure as an $n$-tuple
satisfying a list of axioms, we define the corresponding
structure in CiC. The difference is that the components of an
$n$-tuple are distinguished by the ordering of the components, but the
components (that is fields) of a structure are distinguished by naming
each component.

In a structure declaration, a field marked as {\it parameter} can float out
of the structure and become an unbundled parameter to the structure.
We are relying here on an extension to CiC designed by the author last
year (called Cabarete mode), which facilitates the transformations
between bundled and unbundled structures.

A field marked as {\it type} means that when a value $X$ of that
structure appears in syntax reserved for a type (for example $x : X$
or $f:X \to \alpha$), the value $X$ is coerced to the type field.

A field marked as {\it map} means that when a value $f$ appears in
function application syntax (say $f\ x$), the function $f$ is coerced
to the map field.
\end{remark}

\newpage
\lsection{Set membership}

%- macro -> let_annotation 
In this section let $\alpha,\beta$ be types.

%- macro -> predicate_def 
%- symbol_statement -> LIT_NOT symbol_statement -> LIT_NOT prop
%- The parentheses are optional.
We write $x \ne y$ iff not $(x = y)$.

%- macro -> definition_statement -> classifier_def
Let function, quotient, term, relation be classifiers.


%- definition
\deflabel{set}
%- definition_statement -> type_def 
%- type_word_pattern 
Let \df{set} (of)\ $\alpha$ be the type $\alpha \to\Bool$.
\end{definition}

%-definition
\deflabel{universe}
%- function_def -> lit_lets function_head copula LIT_THE plain_term
%- function_head -> function_word_pattern -> LIT_THE word_pattern
Let the \df{universe} of $\alpha$ be the function
\[
\funmapsto{(\blank : \alpha) : \Bool}{\h{true}}.
\]
\end{definition}

\deflabel{empty set}
%- same parse tree as universe
Let the empty set (inferring $\alpha$) be the function
\[
\funmapsto{(\blank\ : \alpha) : \Bool}{\h{false}}.
\]
\end{definition}

%-macro -> function_def
%- lit_lets symbol_pattern copula LIT_THE plain_term
Let $\emptyset$ stand for the empty set.

\begin{remark}
A set $X$ over $\alpha$ is a subset of the universe of $\alpha$. We
follow our general convention of using the preposition {\it over} with
parameters to a type.

Subsets are identified with their characteristic functions, taking
values in \Prop.  Thus, the universe of $\alpha$ is the constant
function from $\alpha$ to \Prop\ taking value true for each input.

The empty set of $\alpha$ is the other extreme.  It is the constant
function taking value false for each input.

In type theory, there is a universe and an empty set for each type:
the universe of the natural numbers, the empty set of the real
numbers, etc.

In the following, we remark that \h{notation-subset} is a type
isomorphic with \h{binary-relation}.
\end{remark}

\begin{remark}
Notational structures are transient structure that assist in parsing,
but that are not passed through to the kernel for type checking.
Non-transient structure implement notations through field synonyms.

Field labels that are variables are anonymous and unify with any field
of the same type.
\end{remark}

\deflabel{notation in}
%- structure
\h{notation-in} is the notational structure with
\begin{structure}
\firstitem (parameter) $\beta$ % carrier first, to allow dependency.
\item (parameter) $\alpha$
\item \h{notation-in} : $\alpha\to\beta\to \Prop$\texstop
\end{structure}\cnlstop
\end{definition}

\deflabel{in}
We write $x \in X$ (inferring C :
\h{notation-in}) iff $C\h{.notation-in}\ x\ X$.
\end{definition}

\deflabel{notin}
We write $x\notin X$ iff $\h{not}(x\in X)$.
\end{definition}

\deflabel{notation-subset}
\h{notation-subset} is the notational structure with
\begin{structure}
\firstitem (parameter, type) $\alpha$
\item \h{notation-subseteq} : $\alpha \to \alpha \to \Prop$\texstop
%\item subset $X\ (Y : \alpha)\ \assign X \subseteq Y$ and $X\ne Y$
\end{structure}\cnlstop
\end{definition}

\deflabel{subset notation}
We write $X\subset Y$ (inferring $C : \h{notation-subset})$ iff
$C\h{.notation-subset}\ X\ Y$.
\end{definition}

\begin{remark}
When one definition builds on another that has an implicit variable,
it is not necessary to mention the implicit variable again in the new
definition.
\end{remark}


We write $X\subseteq Y$ iff $X\subset Y$ or $(X=Y)$.

We say $X$ is a \df{subset of} $Y$ iff $X\subseteq Y$.

We say $x$ is a \df{member of} $X$ iff $x\in X$.

Let the value of $f$ at $x$ stand for $f(x)$.


\begin{remark}
In the following extension, we pull out the variables $\alpha$ and
$\beta$ for purposes of naming.  The resulting extension is equivalent
to what would be obtained by working with fully bundled structures.

A structure is {\it embedded} (following the terminology of golang) if
all the fields of that structure are inserted into the structure being
declared. The insertion is flat.  That is the fields appear at the top
level (rather than nested) in the structure being declared.
\end{remark}


Moreover, \h{notation-in} over $\beta, \alpha$ implements
\begin{structure}
\firstitem (notation) notation-subset $X\ Y\ \assign\ 
  \h{for all } x,\ x \in X \rightarrow x \in Y$\texstop
\end{structure}\cnlstop

\newpage
\lsection{Natural Number}

\begin{remark}
Here we construct the semiring of natural numbers with its standard
notation. Each inductive type, such as the type of natural numbers,
comes wrapped as an object of type \h{typical}
(with the inductive type as a field marked
\emph{type} to create automatic type coercions to the underlying
inductive type).
\end{remark}

\deflabel{typical structure}
A \h{typical-structure} is a structure with
\begin{structure}
\firstitem (type) $\alpha : \Type$\texstop
\end{structure}\cnlstop
\end{definition}

\deflabel{notation zero}
\h{notation-zero} is the notational structure with
\begin{structure}
\firstitem $\alpha : \Type$
\item $\h{notation-zero} : \alpha$\texstop
\end{structure}\cnlstop
\end{definition}

Let $0$ (inferring
$C:\h{notation-zero}$) stand for $C\h{.notation-zero}$.

\deflabel{notation one}
\h{notation-one} is the notational structure with
\begin{structure}
\firstitem $\alpha : \Type$
\item \h{notation-one} $ : \alpha$\texstop
\end{structure}\cnlstop
\end{definition}

Let $1$ (inferring $C:\h{notation-one}$) stand for $C\h{.notation-one}$.

\deflabel{notation add}
\h{notation-add} is the notational structure with
\begin{structure}
\firstitem $\alpha : \Type$
\item \h{notation-add} $ : \alpha\to\alpha\to\alpha$\texstop
\end{structure}\cnlstop
\end{definition}

\deflabel{notation add}
Let $x + y$ (inferring $C:\h{notation-add}$) stand for $C\h{.notation-add}\ x\ y$ with precedence
$30$ and left associativity.
\end{definition}

\deflabel{notation numeral}
\h{notation-numeral} is the notational structure with
\begin{structure}
\firstitem notation-one
\item notation-add\texstop
\end{structure}\cnlstop
\end{definition}

\begin{remark} A structure that satisfies notation numeral
has automated support for positive numerals
\[
1,2,3,4,\ldots = 1,\quad 1+1,\quad 1+1+1,\quad 1+1+1+1,\ldots
\]
\end{remark}

\deflabel{notation mul}
\h{notation-mul} is the notational structure with
\begin{structure}
\firstitem $\alpha : \Type$
\item \h{notation-mul} $ : \alpha\to\alpha\to\alpha$\texstop
\end{structure}\cnlstop
\end{definition}


\deflabel{notation mul}
Let $x \* y$ (inferring $C:\h{notation-mul}$) stand for $C\h{.notation-mul}\ x\ y$ with precedence
$40$ and left associativity.
\end{definition}

\deflabel{natural numbers}
Let $\Nat$ be the inductive type
\begin{envMatch}
\firstmatchitem \h{zero} &$:\ \Nat$ 
\matchitem \h{succ} &$:\ \Nat\to\Nat$\texstop
\end{envMatch}\cnlstop

\smallskip
Moreover, $\Nat$ implements
\begin{structure}
\firstitem (notation) notation-zero $\assign \h{zero}$
\item (notation) notation-one $\assign \h{succ}\ 0$\texstop
\end{structure}\cnlstop
\end{definition}

\deflabel{natural number addition}
We define $\h{add}\ (m\ n:\Nat) \assign$
\par$\match\ n$ with
\begin{envMatch}
\firstmatchitem $0$ &$\assign$& $m$
\matchitem $\h{succ}\ k$ &$\assign$& $\h{succ}\ (\h{add}\ m\ k)$\texstop
\end{envMatch}\cnlstop
This exists by recursion.
\end{definition}

Moreover, $\Nat$ implements
\begin{structure}
\firstitem (notation) notation-add $\assign\ \h{add}$\texstop
\end{structure}\cnlstop

\deflabel{natural number multiplication}
We define $\h{mul}\ (m\ n :\Nat) \assign$
\par$\match\ n$ with
\begin{envMatch}
\firstmatchitem $0$ &$\assign$& $0$
\matchitem $k+1$ &$\assign$& $\h{succ}\ (\h{mul}\ m\ k  + m)$\texstop
\end{envMatch}\cnlstop
This exists by recursion.
\end{definition}

Moreover, $\Nat$ implements
\begin{structure}
\firstitem (notation) notation-mul $\assign\ \h{mul}$\texstop
\end{structure}\cnlstop


\newpage
\lsection{List}

\begin{remark}
Intuitively, a list is a finite sequence of elements of the same type.
Lists are one of the fundamental inductive types.  This section
introduces lists, basic operations on lists, and notation.
\end{remark}

In this section, let $\alpha$ be a type.

\deflabel{list}
Let $\h{list}\ \h{(of)}\ \alpha$ be the inductive type
\begin{envMatch}
\firstmatchitem $\h{null}\ \optarg{}$ &$:$& $\h{list}$
\matchitem $\h{cons}$ &$:$& $\alpha\to\h{list}\to\h{list}$\texstop
\end{envMatch}\cnlstop
\end{definition}

\deflabel{null list}
Let $\nullbrack$ (inferring $\alpha$) stand for
\[
\h{null} : \h{list}\ {\alpha}.
\]
\end{definition}

\deflabel{cons}
Let $(x : \alpha) \cons (X : \h{list}\ \alpha)$ stand for
\[
\h{cons}\  x\ X.
\]
\end{definition}

We introduce synonyms size/length.

\deflabel{length}
Let $\h{length} : \h{list}\ \alpha\ \to \Nat \assign$ \par\function
\begin{envMatch}
\firstmatchitem $\nullbrack$ &$\assign$& $0$
\matchitem $a \cons A$ &$\assign$& 1 + \h{length}\ A\texstop
\end{envMatch}\cnlstop
This exists by recursion.
\end{definition}

\deflabel{in}
Let $\h{in}\ (x : \alpha) : \h{list}\ \alpha\to \Prop\assign$  \par\function
\begin{envMatch}
\firstmatchitem $\nullbrack$ &$\assign$& \h{false}
\matchitem $a \cons A$ &$\assign$& $(x = a) \lor (\h{in}\ x\ A)$\texstop
\end{envMatch}\cnlstop
This exists by recursion.
\end{definition}

Moreover, list implements
\begin{structure}
\firstitem (notation) notation-in $(x : \alpha)\ (X : \h{list}\ \alpha) \assign \h{in}\ x\ X$
\end{structure}\cnlstop

\begin{remark} The drop function removes the first occurrence
of an element from a list
(but does nothing if the element does not belong to the list).
\end{remark}

\deflabel{drop}
Let $\h{drop}\ (x : \alpha) : \h{list}\ \alpha\to\h{list}\ \alpha \assign$
\par\function
\begin{envMatch}
\firstmatchitem $\nullbrack$ &$\assign$& \nullbrack
\matchitem $a \cons A$ &$\assign$&
$\h{if}\ (x = a)\ \h{then}\ A\ \h{else}\ a \cons (\h{drop}\ x\ A)$\texstop
\end{envMatch}\cnlstop
This exists by recursion.
\end{definition}

\lsubsection{multiset}

\deflabel{nodup}
Let $\h{nodup} : \h{list}\ \alpha\to\Prop \ \assign$
\par\function
\begin{envMatch}
\firstmatchitem $\nullbrack$ &$\assign$& \h{true}
\matchitem $a \cons A$ &$\assign$&
$\h{nodup}\ A \land a \not\in A$\texstop
\end{envMatch}\cnlstop
This exists by recursion.
\end{definition}


\begin{remark} We implement a function that determines whether
one list is a reordering of another list. The function is-permutation
is an equivalence relation on lists.  We are thus able to form
a quotient of lists by this equivalence relation. The resulting
type is the type of multisets, or lists up to permutation.  By
construction a multiset has an enumeration by a list.
\end{remark}

\deflabel{is-permutation}
Let $\h{is-permutation}\ X\ Y : \Prop \assign$
\par\match\ $(X,Y)$ with
\begin{envMatch}
  \firstmatchitem $(\nullbrack,\nullbrack)$ &$\assign$& \h{true}
                  \expandafter\matchitem \onlyTeX{[5pt]}
$(x\cons X,Y)$ &$\assign$&%
$(x\in Y)\land \h{is-permutation}\ X\ (\h{drop}\ x\ Y)$\texstop
\end{envMatch}\cnlstop
This exists by recursion.
\end{definition}

%$
%\left\{
%\begin{tabular}{ll}
%\ifcond\ (x=y)\ \thencond\ \h{is-permutation}\ X\ Y$ \\
%\elsecond\ $(x\in Y)\land \h{is-permutation}\ X\ (y \cons  (\h{drop}\ x\ Y))$.
%\end{tabular}
%\right\texstop
%$


\deflabel{multiset} Let \h{multiset}\ (of)\ $\alpha$ denote the quotient
of $\h{list}\ \alpha$ by 
\h{is-permutation}.  This exists.
\end{definition}

\begin{remark}
When we create a quotient, we automatically generate a constructor
going from the source type to the quotient type.
Here we get
\[
\h{multiset} : \h{list}\ \alpha \to \h{multiset}\ \alpha.
\]

A duplicate~free~multiset over $\alpha$ is an enumeration of a
finite set.
\end{remark}

\deflabel{length}
Let \h{multiset.length}\ $(X : multiset\ \alpha)$ denote the length of
$Y$ for each and every $X= \h{multiset}\ Y$.
This exists and is welldefined.
\end{definition}

Moreover, multiset\ $\alpha$ implements
\begin{structure}
\firstitem (notation) \h{notation-in} $(x : \alpha)$ $(X : \h{multiset}\ \alpha) \assign$ \par
$x\in Y$ for each and every $X = \h{multiset}\ Y$\texstop
\end{structure}\cnlstop

Let \h{multiset.nodup} $(X : \h{multiset}\ \alpha) : \Prop \assign$
\par
\h{list.nodup}\ $Y$ for each and every $X = \h{multiset}\ Y$.

We introduce synonyms element/point/carrier.

\deflabel{duplicate free multiset}
A \h{duplicate~free~multiset} is a structure with
\begin{structure}
\firstitem (parameter) element : \Type
\item support : multiset element
\item length\ $\assign$ the length of the support
\interitem{such that}
\item nodup support\texstop
  %- a proof that the support set is
 %duplicate free.
\end{structure}\cnlstop
\end{definition}




\newpage
\lsection{Finiteness}

In this section, let $\alpha$ be a type.

\begin{remark}
This section relies on results about lists.

Every list is constructed by a finite process of cons'ing.  This
finite process leads to the notion of finite sets and types.

We can specify a bijection from \h{duplicate~free~multiset} to finite sets.

The function takes a simple form, thanks to our notation $\in$ for
multisets, and curly bracket notation for sets.

Once we have a bijection, there are two ways to interpret $\in$ for
\h{duplicate~free~multisets}.  We need compatibility. %XX
\end{remark}

In this section, let $X$ be a duplicate free multiset $\alpha$;
and let $Y$ be a set of $\alpha$.

\deflabel{finite}
We say that $Y$ is
\df{finite} iff there exists $X$ 
such that $Y = \setcomp{x}{x \in \h{support}\ X}$.
\end{definition}

Let an \df{equivalence} $(f:\alpha\to\beta)$ stand for a bijection $f$.

%\deflabel{set of nodup multiset}
We record as identification
\begin{align*}
\funalign
{\h{duplicate~free~multiset}\ \alpha}
& {\ \xrightarrow{\sim}\ } 
{\h{finite set}\ \alpha}\\
{X}&{\ \mapsto\ }{\setcomp{x}{x \in \h{support of}\ X}}.
\end{align*}
This exists and is unique.



\begin{remark}
We give infrastructure support for bijective functions that have been
recorded in the form of coercions in both directions between the
two.

By recording it as an identification, definitions involving
\h{duplicate~free~multiset} automatically extend to finite sets.

For this reason, size and support are now defined on finite
sets.
\end{remark}

\lsubsection{finite types}

\begin{remark}
We extend the preceding notions from sets to types by using the
universe of a type.  

We must enter a namespace here to avoid conflict in notation.
\end{remark}

We enter the namespace \h{finite-type}.

%- macro
We say that $\alpha$ is a \df{finite type} iff the
universe of $\alpha$ is finite.

\begin{definition}
Assume that $\alpha$ is a finite type.  The
\df{size} of $\alpha$ is the size of the universe of
$\alpha$.  This exists and is unique.
\end{definition}

\begin{definition}
Assume that $\alpha$ is a finite type.  The \df{support}
of $\alpha$ is the support of the universe of $\alpha$.  This exists
and is unique.
\end{definition}


\newpage
\lsection{Order}

\deflabel{binary relation}
A \df{binary~relation} is a structure with
\begin{structure}
\firstitem (parameter, type) element : \Type
\item a (map) $relation : element \to element \to \Prop$\texstop
  \end{structure}\cnlstop
\end{definition}

In this section, fix $(R : \h{binary~relation})$; and 
 let $(s\ x\ y\ z : R)$; and 
write $x \le y$ iff $R\ x\ y$.

\deflabel{reflexive} We say $R$ is \df{reflexive} iff for all\ $x,\ x
\le x$.
\end{definition}

\deflabel{transitive} We say $R$ is \df{transitive} iff for
all\ $x\ y\ z,\ x \le y \land y \le z \to x \le z$.
\end{definition}

\deflabel{symmetric} We say $R$ is \df{symmetric} iff for
all\ $x\ y,\ x \le y \to y \le x$.
\end{definition}

\deflabel{preorder} We say $R$ is a \df{preorder} iff $R$ is symmetric
and transitive.
\end{definition}

\deflabel{equivalence relation} We say $R$ is an \df{equivalence
  relation} iff $R$ is reflexive, symmetric and transitive.
\end{definition}

\deflabel{antisymmetric} We say $R$ is \df{antisymmetric} iff for
all\ $x, y,\ x \le y\ \h{and}\ y \le x\ \h{imply}\ x = y$.
\end{definition}

\deflabel{total} Assume that $R$ is a preorder.  We say that $R$ is
\df{total} iff for all\ $x,y,\ x \le y \ \h{or}\ y \le x$.
\end{definition}

\deflabel{poset} We say that $R$ is a \df{poset} iff $R$ is an
antisymmetric preorder.
\end{definition}

Let \df{partially ordered set} stand for poset.

\deflabel{linear order} We say that $R$ is a \df{linear order} iff $R$
is a total poset.
\end{definition}

Let \df{total order} denote linear order.

We introduce synonyms greatest/maximum/top.

\deflabel{greatest element} We say that $y$ is a \df{greatest element
  in} $R$ iff for all\ $x,\ x \le y$.
\end{definition}

We introduce synonyms least/minimum/bottom.

\deflabel{least element} We say that $y$ is a \df{least element} in
$R$ iff for all\ $x,\ y \le x$.
\end{definition}

Let $x < y$ stand for $x \le y$ and $x \ne y$.

\deflabel{maximal element} We say that $y$ is a \df{maximal element}
in $R$ iff there exists no $x$ such that $y < x$.
\end{definition}

\deflabel{minimal element} We say that $y$ is a \df{minimal element}
in $R$ iff there exists no $x$ such that $x < y$.
\end{definition}

\deflabel{irreflexive} We say that $R$ is \df{irreflexive} iff there
exists no $x$ such that $x < x$.
\end{definition}

\deflabel{asymmetric} We say that $R$ is \df{asymmetric} iff for
all\ $x\ y,\ x < y\ \h{implies that not}\ y < x$.
\end{definition}

\deflabel{strict partial order} We say that $R$ is a
\df{strict partial order} iff $R$ is irreflexive, transitive, and
asymmetric.
\end{definition}

Let $S$ be a set of $R$.

\deflabel{upper bound} We say $x$ is an \df{upper bound} of $S$ in $R$
iff $s \le x\ \h{for all}\ s \in S$.
\end{definition}

\deflabel{lower bound} We say $x$ is a \df{lower bound} of $S$ in $R$
iff $x \le s$ for all $s \in S$.
\end{definition}

\deflabel{wellfounded} We say that $R$ is \df{wellfounded} iff for
every $S$ such that $S\ne \emptyset$ there exists a lower bound $s$ of
$S$ in $R$ such that $s\in S$.
\end{definition}

We introduce synonyms least upper bound/join/supremum.

We introduce synonyms greatest lower bound/meet/infimum.

\deflabel{least upper bound} We say $x$ is a \df{least upper bound} of
$S$ in $R$ iff $x$ is an upper bound of $S$ such that for every upper
bound $y$ of $S$ we have $x \le y$.
\end{definition}

\deflabel{greatest lower bound} We say $x$ is a \df{greatest lower
  bound} of $S$ in $R$ iff $x$ is a lower bound of $S$ such that for
every lower bound $y$ of $S$ we have $y \le x$.
\end{definition}

\deflabel{join semilattice} We say $R$ is a \df{join semilattice} iff
for every $x, y$, there exists a least upper bound of $\setenum{x,y}$ in
$R$.
\end{definition}

\deflabel{meet semilattice} We say $R$ is a \df{meet semilattice} iff
for every $x, y$, there exists a greatest lower bound of $\setenum{x,y}$ in
$R$.
\end{definition}

\deflabel{join} Assume $R$ is a join semilattice.  Let
$\h{\df{join}}\ x\ y$ be the least upper bound of $\setenum{x,y}$ in $R$.
This exists and is unique.
\end{definition}

\deflabel{meet} Assume $R$ is a meet semilattice.  Let
$\h{\df{meet}}\ x\ y$ be the greatest lower bound of $\setenum{x,y}$ in $R$.
This exists and is unique.
\end{definition}

\deflabel{lattice} $R$ is a \df{lattice} iff $R$ is a join semilattice
and a meet semilattice.
\end{definition}

\newpage
\lsection{Order Notation}

\deflabel{notation le}
\h{notation-le} is the notational structure with
\begin{structure}
\firstitem (type) $\alpha : \Type$
\item \h{notation-le} : $\alpha \to \alpha \to \Prop$\texstop
%\item lt $x\ (y : D)\ \assign x \le y$ and $x\ne y$
%\item ge $x\ (y : D)\ \assign y \le x$
%\item gt $x\ (y : D)\ \assign y < x$
%where $D = $\h{element-le}
\end{structure}\cnlstop
\end{definition}

We write $x \le y$ (inferring
$C:\h{notation-le}$) iff $C\h{.notation-le}\ x\ y$.


We write $x < y$ iff $x \le y$ and $x\ne y$.

We write $x \ge y$ iff $y \le x$.

We write $x > y$ iff $y < x$.

We say $m$ is \df{at most} $n$ iff $m \le n$.

We say $n$ is \df{at least} $m$ iff $n \ge m$.

We say $m$ is \df{less than} $n$ iff $m < n$.

We say $n$ is \df{greater than} $m$ iff $n > m$.

\newpage
\lsection{More on Natural Numbers}

In this section, let $m,\ n,\ d,\ p,\ r$ be natural numbers.

\deflabel{nat le}
  We write that $\h{nat.nat-le}\ m \ n$ iff there exists $d$ such that $m + d = n$.
\end{definition}

Moreover, $\Nat$ implements
\begin{structure}
\firstitem (notation) notation-le $\assign$ nat-le\texstop
\end{structure}\cnlstop

%We record a wellfounded total order, where
%\begin{structure}
%\firstitem relation\ $\assign \h{nat-le}$
%\item (notation) \h{notation-le} $\assign$ relation
%\end{structure}\cnlstop
%This exists and is unique.


\begin{remark}
The numbers
$0,\pm1,\pm2,\ldots$ can be interpreted in any structure that
additionally satisfies \h{notation-neg}.
\end{remark}

We introduce synonyms number\plural s.

Let \df{natural number} stand for \Nat.

\deflabel{divides}
We say that $d$ \df{divides} $m$ iff $d\ne 0$ and there exists $r$
such that $m= d\*r$.
\end{definition}


We write $\natdiv{d}{m}$ iff $d$ divides $m$.

We say that $d$ is a \df{divisor of} $m$ iff $d$ divides $m$.


\deflabel{natpow}
Assume that $b,n$ are natural numbers.  We define
$\natpow{b}{n}\assign$
\par$\match\ n$ with
\begin{envMatch}
\firstmatchitem $0$ &$\assign$& $1$
\matchitem $m + 1$ &$\assign$& $b \* \natpow{b}{m}$\texstop
\end{envMatch}\cnlstop
This exists by recursion.
\end{definition}

\deflabel{prime}
Assume that $p$ is a natural number greater than $1$.  We say that $p$
is a \df{prime} iff each divisor of $p$ is equal to $1$ or is equal to
$p$.
\end{definition}

\deflabel{multiplicity}
Assume that $n$ is a positive natural number.
The \df{multiplicity} of $p$ in $n$ is the natural number $m$ such that
\[
\natpow{p}{m}\ \h{divides}\ n\quad \h{and}\quad \natpow{p}{m+1}\ %
\h{does not divide}\ n.
\]
This exists and is unique.
\end{definition}

\newpage
\lsection{Group theory}

In this section, let $\alpha$ be a type.

\deflabel{magma}
A \h{magma} is a structure with
\begin{structure}
\firstitem (type, parameter) element : \Type
\item an $\h{op} : \h{element} \to \h{element} \to \h{element}$
\item (notation) $\h{notation-mul} \assign op$\texstop
\end{structure}\cnlstop
\end{definition}


%We record a coercion 
%\begin{align*}
%\funalign
%{\h{magma}\ \alpha}{\ \to \ }{\h{notation-in}}
%{M}{\ \mapsto \ }
%\$(G : group) : \h{notation-in}\assign$
%{
%\Make\begin{make}
%\firstitem \h{notation-in} $\beta\assign \h{group}$
%\item $\alpha\assign M\h{.element}$\texstop
%\end{make}\cnlstop
%}
%\end{align*}


\deflabel{abelian}
Assume that $M$ is a magma.
M is \df{abelian} iff for all $(x\ y : M)$, $x \* y = y \* x$.
\end{definition}

\deflabel{semigroup}
A \df{semigroup} is a structure with
\begin{structure}
\firstitem a magma
\interitem{such that}
\item for all elements $x,\ y,\ z$,\ 
x \* y \* z = (x \* y) \* z\texstop
\end{structure}\cnlstop
\end{definition}

% for all $(x, y, z : element), x \* y \* z = (x \* y) \* z$.
% for all $x$, $y$ and  $z$, $x \* y \* z = (x \* y) \* z$.
% for all $(x,\ y,\ z : \h{element}),\ x \* y \* z = (x \* y) \* z$.
% for all $x$, $y$ and  $z$, $x \* y \* z = (x \* y) \* z$.

\deflabel{monoid}
 A \df{monoid} is a structure with
\begin{structure}
\firstitem a semigroup
\item a unit: element
\item (notation) $\h{notation-one} \assign \h{unit}$
\interitem{such that}
\item for all\ $x,\ x \* 1 = 1 \* x = x$\texstop
\end{structure}\cnlstop
\end{definition}

\deflabel{notation inverse}
\h{notation-inverse} is the notational structure with
\begin{structure}
\firstitem (parameter) $\alpha : \Type$
\item \h{notation-inverse} : $\alpha\to\alpha$\texstop
\end{structure}\cnlstop
\end{definition}


Let $\inv{x}$ (inferring
$C:\h{notation-inverse}$) stand for $C\h{.notation-inverse}\ x$.


\deflabel{group}
A \df{group} is a structure with
\begin{structure}
\firstitem a monoid
\item an inv : $C \to C$ 
\interitem{such that}
\item (notation) $\h{notation-inverse} \assign \h{inv}$
\item $\forall (x : C),\  x \* \inv{x} = \inv{x} \* x = 1$\texcomma
\end{structure}
\wherearg{C \assign\h{element}}.
\end{definition}

\newpage
\lsection{Ring and modules}

\begin{remark}
An additive group is just a group in which the surface notation has
been altered.  Additive group and group satisfy each other (but the notational
fields drop out).
%The convention is that 'xxx extensions keep field names distinct,
%but satisfaction ignores 'xxx extensions, but gives priority to fewer ''.
\end{remark}


\deflabel{notation neg}
\h{notation-neg} is the notational structure with
\begin{structure}
\firstitem $\alpha : \Type$
\item $\h{notation-neg} : \alpha\to\alpha$\texstop
\end{structure}\cnlstop
\end{definition}

Let $-x$ (inferring
$C:\h{notation-neg})$
stand for $C\h{.notation-neg}\ x$.


\deflabel{additive group}
An \df{additive group} is a structure with
\begin{structure}
\firstitem (notationless)  group 
\item (notation) $\h{notation-zero} \assign \h{unit}$
\item (notation) $\h{notation-add} \assign \h{op}$
\item (notation) $\h{notation-neg} \assign \h{inv}$\texstop
\end{structure}\cnlstop
\end{definition}


\deflabel{ring}
A \df{ring} is a structure with
\begin{structure}
\firstitem an abelian additive group
\item an $\h{op'}  : C \to C \to C$
\item a $\h{unit'} : C$
\item (notation) $\h{notation-mul} \assign \h{op'}$
\item (notation) $\h{notation-one} \assign \h{unit'}$
\interitem{such that}
\item $\forall\ (x : C),\ 1\*x = x\*1 = x$
\item $\forall\ (x\ y\ z : C),\ x\*y\*z = x\*(y\*z)$
\item $\forall\ (x\ y\ z\ : C),\ (x + y)\*z = x\*z + y\*z$
\item $\forall\ (x\ y\ z : C),\ z\*(x+y) = z\*x + z\*y$\texcomma
\end{structure}
\wherearg{C \assign \h{element}}.
\end{definition}


\begin{remark} 
By construction, a ring satisfies a group (with additive structure).
There is a separate map from ring to monoid (with multiplicative
structure), which is achieved with an explicit function call.
\end{remark}


\deflabel{multiplicative monoid of ring}
Let $\h{monoid-of-ring}\ ($R$ : \h{ring}) : \h{monoid} \assign$
\Make \begin{make}
%\item element \assign R\h{.element}
\firstitem unit $\assign R\h{.unit'}$
\item op $\assign R\h{.op'}$
\item \etc\texstop
\end{make}\cnlstop
\end{definition}


\deflabel{commutative} 
Assume that $R$ is a ring.  We say that $R$ is
\df{commutative} iff for every $(x\ y : R),\ x\*y = y\*x$.
\end{definition}


\deflabel{nontrivial}
Assume that $R$ is a ring.
We say that $R$ is \df{nontrivial} iff $(0 : R) \ne (1 : R)$.
\end{definition}


\deflabel{unit} 
Assume that $R$ is a commutative ring.  Assume that
$(x : R)$.  We say that $x$ is a \df{unit} in $R$ iff there exists $(y
: R)$ such that $x \* y = y \* x = 1$.
\end{definition}


\deflabel{notation dot}
\h{notation-dot} is the notational structure with
\begin{structure}
\firstitem (parameter, type) $\alpha, \beta : \Type$
\item (notation) \h{notation-dot} : $\alpha \to \beta \to \beta$\texstop
\end{structure}\cnlstop
\end{definition}

Let $r \cdot x$ (inferring $C:\h{notation-dot}$)
stand for $C\h{.notation-dot}\ r\ x$.


\deflabel{left module}
A \df{left module} is a structure with
\begin{structure}
\firstitem (parameter) $R$ : ring
\item an  additive group
\item a \h{scalar-multiplication} : 
$R\h{.element} \to$ element $\to$ element
\item (notation) $\h{notation-dot} \assign \h{scalar-multiplication}$
\interitem{such that}
\item for all $r,x,y,\ r \cdot (x + y) = r \cdot x + r \cdot y$
\item 
for all $r,s,x,\ (r + s) \cdot x = r \cdot x + s \cdot x$
\item 
for all $r,s,x,\ (r\*s) \cdot x = r \cdot (s \cdot x)$
\item for all $x,\ 1 \cdot x = x$\texstop
\end{structure}\cnlstop
\end{definition}

\newpage
\lsection{The field of real numbers}

\deflabel{field}
A \df{field} $R$ is a nontrivial commutative ring such that for all $x\ne 0$,  $x$ is a unit in $R$.
\end{definition}

\deflabel{vector space}
  A \df{vector space} is a structure with
  \begin{structure}
\firstitem a left module
\item (parameter) $R$ : field\texstop
  \end{structure}\cnlstop
\end{definition}


\deflabel{ordered ring}
An \df{ordered ring} is a structure with
\begin{structure}
\firstitem a ring
\item a linear order 
\interitem{such that}
\item (notation) $\h{notation-le} \assign$ relation
\item for all $x,y,z$, if $x \le y$, then $x + z \le y + z$
\item for all $x,y$, if $0\le a$ and $0\le b$, then $0\le a\*b$\texstop
\end{structure}\cnlstop
\end{definition}

%See \url{https://en.wikipedia.org/wiki/Ordered-ring}


\deflabel{ordered field}
An \df{ordered field} is a a structure with
\begin{structure}
\firstitem an  ordered ring
\item a field\texstop
\end{structure}\cnlstop
\end{definition}

\deflabel{complete}
Assume that $F$ is an ordered field.
We say that $F$ is \df{complete} iff
for every subset $S$ of the universe of $F$,
  if $S \ne \emptyset$ and $S$ has an upper bound in $F$, then $S$ has a least upper bound in $F$.
\end{definition}

\deflabel{absolute value}
Let
$\h{ordered-field.absolute-value}\assign$
%\ (x : \h{element})\assign$ 
\begin{align*}
\funalign
{\h{element}}&{\quad\to \ }{\h{element}}\\
{x}&{\quad\mapsto\ }
{
\begin{cases}
\firstmatchitem \caseif{\phantom{-}x}{0 \le x}
\matchitem \caseotherwise{-x}\texstop
\end{cases}
}
\end{align*}
\cnlstop
This exists and is unique and total.
\end{definition}

In this section, let $(F\ G : \h{complete order field})$.

\deflabel{isomorphism of complete ordered fields} 
An isomorphism of complete ordered fields  from $F$ to $G$  is a bijection
\[
f:F\to G
\]
such that 
\begin{align*}
&\h{for all}\ x, y,\ f(x+y) = f(x) + f(y) \land f(x\* y) = f(x)\* f(y)\quad \land \\
&\h{for all}\ x, y,\ x\le y\ \iff\ f(x) \le f(y).
\end{align*}
\end{definition}

We record as identification
the isomorphism of complete ordered fields
from $F$ to $G$.  This exists and is unique.

\begin{remark}
An identification allows us to automatically transport
structure from $F$ to $G$.
\end{remark}

\deflabel{real}
Let \Real\ be the complete ordered field.
This exists and is canonical.
\end{definition}

\lsubsection{norms}

Let $\realabs{(x:\Real)}$ denote \h{absolute-value}\ $x$.

\deflabel{normed commutative ring}
  A \df{normed commutative ring} is a structure with
  \begin{structure}
\firstitem a commutative ring
\item an \h{absolute-value} : element $\to \Real$
\interitem{such that}
\item for all $x$, $A\ x = 0 \iff x = 0$
\item for all $x\ y,\ A (x + y) \le A\ x + A\ y$
\item for all $x\ y, A(x \* y) \le (A\ x) \* (A\ y)$\texcomma
  % See https://ncatlab.org/nlab/show/normed+ring
\end{structure}
\wherearg{A \assign \h{absolute-value}}.
\end{definition}

\deflabel{normed field}
  A \df{normed field} is a structure with
  \begin{structure}
\firstitem a normed commutative ring
\item a field\texstop
  \end{structure}\cnlstop
\end{definition}



\newpage
\lsection{Sylows Theorems}

\lsubsection{meet semilattice of subgroups}

\begin{remark}
We introduce the semilattice of subgroups explicitly.  Each subgroup
is characterized by its set of elements.  A subgroup is not literally
a group.  It would take a coercion to make it so.

Because of the way parameters work, we say that $H$ is a subgroup {\it
  over} $G$.
\end{remark}

\deflabel{subgroup}
A \df{subgroup}  is a structure with
\begin{structure}
\firstitem (parameter) \h{ambient-group} : group
\item (type) support $: \h{set of}\ \h{ambient-group}$
\interitem{such that}
\item $1 \in support$ and
\item for all $x,y \in \h{support}, x \* \inv{y} \in \h{support}$\texstop
\end{structure}\cnlstop
\end{definition}


\deflabel{subgroup semilattice} Let 
\h{subgroup-semilattice}
$(G :\h{group}) : \h{meet-semilattice}\assign$
\Make \begin{make}
\firstitem $\h{element}\ \assign \h{subgroup over}\ G$
\item $\h{relation}\ H_1\ H_2 \assign \h{for all}\ x,\ x \in H_1 \to x \in
H_2$
\item \etc\texstop
\end{make}\cnlstop
This exists and is unique.
\end{definition}

\begin{remark} At this point, we could introduce notation
$\subseteq$, $\cap$, etc. for subgroups over a given group.

In the following definition of subgroup order, the parentheses around
(subgroup) indicate that the qualifier \emph{subgroup} appears in the
\LaTeX\ file, but is not displayed in the pdf.
\end{remark}

Assume that $G$ is a finite group, let
the \df{order} of $G$ stand for the size of $G$.

We enter the namespace subgroup.

\deflabel{subgroup order}
Assume that $H$ is a subgroup over a finite group $G$.
The \df{order} of $H$ is the size of the support of $H$.
This exists and is well propped \parenthetical{that is, $H$ is finite}.
\end{definition}


\lsubsection{statement}

In this section, let $G$ be a fixed finite group.


In this section, let $\conj{g}{X}$ stand for
\[
\setcomp{g\*x\*\inv{g}}{x\in X},
\]
\where\ $(g : G)$ $(X : \h{set}\ G)$.

\deflabel{conjugate}
Assume that $(g : G)$.  Assume that $H$ is a subgroup over $G$.  The
\df{conjugate} of $H$ by $g$ in $G$ is the subgroup $\conj{g}{H}$ over $G$.
\end{definition}

\deflabel{normalizer}
Assume that $H$ is a subgroup over $G$.  The \df{normalizer of} $H$
\df{in} $G$ is the subgroup $N$ over $G$ such that for all $x$, $x \in N
\iff \conj{x}{H}=H$. This
exists and is unique.
\end{definition}


Let $\grouporder{G}$ denote the order of $G$.

In this section, let $p$ be a fixed prime number.


In this section, let $\mult{p}{G}$ denote the multiplicity of $p$ in
$\grouporder{G}$.


\deflabel{Sylow}
A \df{Sylow $p$ subgroup of} $G$ is a subgroup $P$ over $G$ such that
the order of $P$ is $\natpow{p}{\mult{p}{G}}$.
\end{definition}


\begin{definition}
Let $\Syl{p}{G} \assign 
\setcomp{P}{(P\ \h{is a Sylow}\ p\ \h{subgroup of}\ G)}$.
\end{definition}



Let $\SylN{p}{G}$ denote the size of $\Syl{p}{G}$.  This is well
propped \parenthetical{that is, there are finitely many Sylow $p$
  subgroups}.



\begin{definition}
Let $\Nz{p}{G}$ denote
the size of the normalizer of each and every Sylow $p$ subgroup of $G$.
This exists, is unique and is well-defined.
\end{definition}

\begin{theorem}[Sylow1]
There exists a Sylow $p$ subgroup of $G$.
\end{theorem}

\begin{theorem}[Sylow 2]
If $P, P'$ are Sylow $p$ subgroups of $G$, then there exists $(g : G)$
such that $P' = \conj{g}{P}$.
\end{theorem}

\begin{theorem}[Sylow 3a]
Assume that $\grouporder{G} = p'\*\natpow{p}{\mult{p}{G}}$.
We have $\SylN{p}{G}$ divides $p'$.
\end{theorem}

\begin{theorem}[Sylow 3b]
We have $p$ divides $(\SylN{p}{G} - 1)$.
\end{theorem}

\begin{theorem}[Sylow 3c]
We have $\SylN{p}{G}\*\Nz{p}{G} = \grouporder{G}$.
\end{theorem}

\end{cnl}


\end{document}
