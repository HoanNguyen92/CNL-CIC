% Inititalization of CNL commands.
% This file should be read inside \begin{cnl}...\end{cnl}
% 

%- cnlinit.tex file 

%- SECTIONS

\CnlCustom[1]\parenI{ (#1) }
\CnlCustom\*{*}
\CnlCustom[1]\lsection{Section \concat{}{#1} .}
\CnlCustom[1]\lsubsection{Subsection \concat{}{#1} .}
\CnlCustom[1]\lsubsubsection{Subsubsection \concat{}{#1} .}
\CnlCustom[1]\deflabel{\begin{definition}\label{#1}}
\CnlCustom[1]\df{ #1 }
\CnlCustom[1]\h{ #1 }

%- CUSTOM WORDS

\CnlCustom\where{where}
\CnlCustom\ifcond{if}
\CnlCustom\otherwise{true}
\CnlCustom\elsecond{else}
\CnlCustom\where{where}
\CnlCustom\Prop{Prop}
\CnlCustom\Type{Type}
\CnlCustom\Bool{Bool}
\CnlCustom\iff{iff}
\CnlCustom\match{match}
\CnlCustom\function{function}
\CnlCustom\Make{make}
\CnlCustom\Nat{Nat}
\CnlCustom\Real{Real}
\CnlCustom\quot{quot}
\CnlCustom\wherearg{where}
\CnlCustom\fun{fun}


\CnlCustom\assign{ \concat{}{:=} }
\CnlCustom\cons{\concat{}{::}} 
\CnlCustom\plural{\concat{}{/-}}
\CnlCustom[2]\funmapsto{(fun #1 \assign #2)}
%\funalign{A}&{\to}{B}\\{a}{\mapsto}{b}
\CnlCustom[6]\funalign{(fun (#4 : #1) \assign (#6 : #3))}
\CnlCustom\optarg{}
\CnlCustom\etc{\_}
\CnlCustom\prefix{\concat}
\CnlCustom\cnlstop{.}
\CnlCustom\matchitem{\alt}
\CnlCustom\firstmatchitem{\alt}
\CnlCustom[2]\caseif{#2 \assign #1}
\CnlCustom[1]\caseotherwise{true \assign #1}
\CnlCustom\rightarrow{\imply}
%\CnlCustom\nullbrack{\concat{}{[]}}
%\CnlNoExpand[1]\section



% Prohibited control sequences. 
% That is, they should remain outside the cnl environment.
% We should list many TeX primitive control sequences here.

%- CNLERROR

\CnlError\if
\CnlError\else
\CnlError\fi
\CnlError\let
\CnlError\futurelet
\CnlError\afterassignment
\CnlError\usepackage


%- CNLDELETE

\CnlDelete\expandafter
\CnlDelete[1]\interitem
\CnlDelete\texstop
\CnlDelete[1]\onlyTeX
\CnlDelete[1]\phantom
\CnlDelete\firstitem
\CnlDelete\texstop
\CnlDelete\texcomma
% math modes and space
\CnlDelete\ensuremath
\CnlDelete\text
\CnlDelete\thinmuskip
\CnlDelete\medmuskip
\CnlDelete\thickmuskip
\CnlDelete\quad
\CnlDelete\qquad
\CnlDelete\,
\CnlDelete\:
\CnlDelete\;
\CnlDelete\!
\CnlDelete\ %space
\CnlDelete\enspace
\CnlDelete[1]\hspace
\CnlDelete\hfil
\CnlDelete\hfill
\CnlDelete\thinspace
\CnlDelete\left
\CnlDelete\right
\CnlDelete\big
\CnlDelete\Big
\CnlDelete\bigg
\CnlDelete\Bigg
\CnlDelete\allowdisplaybreaks
%other  spacing
\CnlDelete\noindent
\CnlDelete\indent
\CnlDelete[1]\vspace
\CnlDelete\null
\CnlDelete\break
\CnlDelete\newline
\CnlDelete\newpage
\CnlDelete\vfil
\CnlDelete\vfill
\CnlDelete\smallskip
\CnlDelete\medskip
\CnlDelete\bigskip
\CnlDelete[2]\rule
\CnlDelete[1]\parenthetical

%- IGNORED ENVIRONMENTS 

\CnlEnvirDelete{remark}
\CnlEnvirDelete{summary}
\CnlEnvirDelete{tikzpicture}
\CnlEnvirDelete{fancyvrb}

