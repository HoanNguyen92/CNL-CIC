\documentclass[12pt]{amsart}


% PACKAGES
\usepackage{url}
\usepackage{amsmath}
\usepackage{amsthm}
\usepackage{amssymb}

% for underscores https://texfaq.org/FAQ-underscore
\usepackage{lmodern}
\usepackage[T1]{fontenc}
\usepackage{textcomp}
\usepackage{lineno}

\usepackage[
bookmarksopen,
bookmarksdepth=2,
%breaklinks=true
colorlinks=true,
urlcolor=blue]{hyperref}

% GLOBAL FORMATTING
%\linenumbers
\parindent=0pt
\parskip=0.5\baselineskip
\raggedbottom

% TITLE AUTHOR DATE
\title{Controlled natural language for type theory}

\date{November 17, 2019}
\author{Thomas Hales}

% THEOREMS 
\newtheorem{definition}{Definition}
\newtheorem{theorem}[definition]{Theorem}
\newtheorem{lemma}[definition]{Lemma}
\newtheorem{specification}[definition]{Specification}

% COMMANDS
\renewcommand{\iff}{\leftrightarrow}
\newcommand{\Prop}{\text{\tt Prop}}
\newcommand{\Type}{\text{\tt Type}}
\newcommand{\fld}{\textasciicircum}
\newcommand{\dequiv}{\mathrel{:=}} %{\mathrel{:\equiv}}
\newcommand{\Nat}{\ensuremath{{\mathbb N}}}
\newcommand{\Real}{\ensuremath{{\mathbb R}}}
\newcommand{\df}[1]{\text{\bf #1}}
\newcommand{\h}[1]{\text{#1}}
\newcommand{\join}{\lor}
\newcommand{\Mid}{\mathrel{\|}}
\newcommand{\comment}[1]{\%- \nobreak{#1}}
\renewcommand{\~}{\ }
\newcommand{\ignore}[1]{}
%\newcommand{\remark}[1]{(#1)}
\renewcommand{\_}{\textunderscore}
\renewcommand\labelitemi{-}
\renewcommand{\qed}{\ensuremath{\square}}

% ENVIRONMENTS

% \leavevmode\par is to make remark work when it is the first item in a subsection.
\newenvironment{remark}
{\leavevmode\par\begin{tabular}{|p{13cm}}\parskip=\baselineskip{\bf Remark.}}
{\end{tabular}}

\newenvironment{oblongo}{}{}

\newenvironment{prule}%
               {\begin{itemize}}%
               {\end{itemize}}
\newcommand{\ptem}{\item}
\newcommand{\nt}[1]{{\tt #1}}
\newcommand{\rw}{$\quad\to\quad$}



% DOCUMENT

\begin{document}
\maketitle

\section{Introduction}

This abstract describes the current designs and development of a
controlled natural language for mathematics that compiles to the Lean
proof assistant.  We call this language Colada (short for
\emph{Co}ntrolled \emph{la}nguage \emph{da}ta).

The design of language grows out of previous controlled natural
languages for mathematics (specifically, Forthel-Naproche-SAD), as
described in Peter Koepke's AITP 2019 talk.  We use
Forthel-Naproche-SAD (or simply Forthel) as a generic name for any of
the dialect inspired by Forthel, and our language Colada is one of
those dialects. This document will refer to the Colada language as
\emph{our dialect}.

Documents in our dialect are written in a specially prepared \LaTeX\ file.
The output of the file will eventually be a type-checked Lean file,
although parts of this process are still under development.  (Checking
mathematical proofs is not currently part of our intended project;
type-checked Lean we exclude proof-checking.)

\subsection{Controlled Natural Languages (CNL)}\label{sub:CNL}

By controlled natural language for mathematics (CNL), we mean an
artificial language for the communication of mathematics that is (1)
designed in a deliberate and explicit way with precise
computer-readable syntax and semantics, (2) based on a single natural
language (which for us will be  English), and (3) broadly
understood at least in an intuitive way by mathematically literate
speakers of the natural language.

CNLs can achieve a much higher degree of English fluency than other
proof-checking languages such as the \emph{Mizar} and DeBruijn's
\emph{vernacular}.  Some other proof languages are purely stylistic,
such as \emph{structured derivations}.  

At AITP 2019 Peter Koepke displayed a short proof from Rudin's {\it
  Principles of mathematical analysis} that he modified with Steffen
Frerix so that it can be read and checked by their system. Their
modified proof is written in fluent English, is typeset by \LaTeX, and
yet is fully checkable. (The target of their language is first-order
logic.)

It is our belief that controlled natural languages are undervalued
technologies in AITP.  Following a divide-and-conquer strategy, our
basic aim is to develop a technology that lies roughly at the midpoint
between current practice of research mathematicians and the current
practice within the proof assistant community.  

\subsection{Lean}

Lean theorem prover is a proof assistant built on the
foundations of calculus of inductive constructions.

The eventual target of our CNL is Lean.

Why is Lean the target?  We wanted something more powerful than 
first-order logic as used in other Forthel dialects.
Many mathematicians are finding it a good system
for research-level mathematics.  There is also the M.K. argument
that if we succeed in bridging the gap between English and Lean,
then automated translation tools will eventually give us translations
from Lean to other proof assistants.


\subsection{Research to Date}

This abstract describes the current stage of a project that is
intended to continue over a period of years.

Our specific research contributions to date are as follows.

\subsubsection{A design and specification of a controlled natural
  language}

Like other Forthel dialects, our grammar is not a context-free.
However, it is similar to a context-free grammar by being described as
a collection of terminal and nonterminal symbols and production rules.
However, the grammar grows as a document is parsed by the addition of
new production rules, which are described in the document.

Our dialect can be viewed as a fusion of three different syntaxes:
Forthel-Naproche-SAD syntax, \LaTeX\ syntax, and Lean theorem-prover
syntax.  

The lexical structure of our dialect is specified in sedlex, a lexical
generator tool for OCaml.

Our dialect has been specified in menhir, an OCaml-based
parser-generator tool for LR(1) grammars.  (Although our dialect is
not an LR(1) grammar, which prevents menhir from automatically
generating a parser, the software checks that we have a well-formed
grammar.)

Our grammar is both complex and recursive to an extraordinary degree.
The grammar has about 350 nonterminals and about 700 production
rules.  The grammar has about 150 context-dependent key words.  (This
is before any of the user-defined grammar extensions.)  However, we
feel that a some complexity is justified (and even required) to
capture widespread mathematical idioms and formulas, the syntax of type
theory (for us, the calculus of inductive constructions), and their
interactions.

We keep most features of Forthel, such its handling of synonyms, noun
phrases, verbs, and adjectives; and its grammar extension mechanisms.
We add additional features such as operator precedence parsing (with
user-specified precendence levels and associativities); scoping;
syntax for \LaTeX\ macros; and dependent type theory including
inductive types and mutual inductive types, structures, and lambda
terms.

A parser for our grammar has been implemented in OCaml, building on
the parser combinator library that John Harrison wrote to parse HOL
Light.  However, this parser is still at an early stage. For example,
it is still not capable of transforming a syntax tree of a parsed
document into expressions that can be processed Lean. (Building this
capability will be a major project.)

We try to make the grammar unambiguous by always taking the longest
match possible in a greedy way.  By greedy, we mean that we take
the longest match of a given production rule when it occurs 
even if doing so produces a match for the enclosing production rule 
of suboptimal length. Whenever this is not the desired behavior,
the author of the document must insert parentheses.

And if, for example, a symbol has been assigned several meanings, it
is always the most recently declared that wins out.


\subsubsection{\TeX to raw controlled natural language syntax}


We have written a software program takes a specially prepared
\LaTeX\ file as input and strips away the non-semantic content and
outputs raw (such as headers, spaces and other layout, graphics,
remarks, and dollar signs).

To create a specially prepared \LaTeX\ file, the author imports a
package \verb!\usepackage{cnl}!. Then any text that appears within the
\emph{cnl} \LaTeX\ environment is handled as controlled natural
language.  Arbitrary \LaTeX\ and graphics may appear outside the CNL
environment.  The key to beautifully typeset \TeX\ documents is an
dual expansion system for macros.  The \TeX\ engine expands macros in
the usual way, but the CNL engine expands some macros according to an
independent semantic specification.

\section{examples}

One project we have started is the translation of the number theory
Planet Math files to our CNL dialect.

We have been using a 30-page CNL file (for debugging) that 
gives many elementary definitions of groups, rings, fields, real numbers, etc.
It goes as far as statements of Sylow's theorems in group theory.


\section{motivation and summary}

This is part of the Formal Abstracts project.

A CNL fixes a problem we encountered early on in the Formal Abstracts
project.   If we leave out proofs, we need more eyes, but Lean
is not written for human eyes (or only as a secondary objective).  
Lean is optimized for writing 
computer-checked proofs.  
One can often not tell what the idea of a proof is by reading a Lean file.
Therefore we need a language that all mathematicians can read.

Our language is already finding uses at this early stage of
development.  For example, our tool can reveal to the author of a
\LaTeX\ document which terms and symbols are being used without
definition.  Ultimately each symbol, word, and phrase can be traced
its source either as a keyword of our dialect or as author-supplied.

I thank the Thang Long University Formal Abstracts team for feedback,
in their use of the software.  I thank Jesse Han, who wrote a
preliminary version of the parser using Haskell's megaparsec parser
combinator library.  I thank Peter Koepke for inspiring this project
and for many conversations.  I thank many others at the project github page.

bibliography

Lean proof assistant.

CiC.

My blog post on CNL and github.

Koepke AITP and github
\footnote{\href{http://aitp-conference.org/2019/aitp19-proceedings.pdf}{AITP
    2019 proceedings}, page 84}

Paskevich thesis and paper

Glushkov and history.

\end{document}
