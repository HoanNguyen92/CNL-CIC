\documentclass[12pt]{amsart}


% PACKAGES
\usepackage{url}
\usepackage{amsmath}
\usepackage{amsthm}
\usepackage{amssymb}
\usepackage{fancyvrb}

% for underscores https://texfaq.org/FAQ-underscore
\usepackage{lmodern}
\usepackage[T1]{fontenc}

\usepackage{textcomp}


\usepackage[
bookmarksopen,
bookmarksdepth=2,
%breaklinks=true
colorlinks=true,
urlcolor=blue]{hyperref}

% GLOBAL FORMATTING
%\linenumbers
\parindent=0pt
\parskip=0.5\baselineskip
\raggedbottom

% TITLE AUTHOR DATE
\title{Notes on translating \TeX\ to a controlled natural language}

\date{July 17, 2019}                                           % Activate to display a given date or no date
\author{Thomas Hales}

% THEOREMS 
\newtheorem{definition}{Definition}
\newtheorem{theorem}[definition]{Theorem}
\newtheorem{lemma}[definition]{Lemma}
\newtheorem{specification}[definition]{Specification}

% COMMANDS
\renewcommand\labelitemi{-}


% ENVIRONMENTS

% \leavevmode\par is to make remark work when it is the first item in a subsection.
\newenvironment{remark}
{\leavevmode\par\begin{tabular}{|p{13cm}}\parskip=\baselineskip{\bf Remark.}}
{\end{tabular}}

\newenvironment{oblongo}{}{}

\newenvironment{prule}%
               {\begin{itemize}}%
               {\end{itemize}}
\newcommand{\ptem}{\item}
\newcommand{\nt}[1]{{\tt #1}}
\newcommand{\rw}{$\quad\to\quad$}



% DOCUMENT

\begin{document}
\maketitle



\section{\TeX-CNL translation specification}

The notes here add detail to the
 \href{https://github.com/formalabstracts/CNL-CIC/wiki/Conversion-from-LaTeX}{CNL-CIC Wiki}.

A \TeX\ file written in CNL-enabled \TeX\ can be processed 
by the \TeX\ engine or by the CNL engine.

%A boolean flag {\tt isPresentation} in the \TeX\ source determines
%which mode is used for processing.

When
we talk about the pdf display, we mean the pdf output 
from \TeX.

The CNL image
means the output of the \TeX\ to CNL translation.  The CNL image is a
text file written in the CNL grammar.


\subsection{Differences in CNL-enabled files}

There are some visible differences between a typical \TeX\ source
and a CNL-enabled one.

\subsection{Variables and Identifiers}

Identifiers can appear with the underscore character.  They must be
processed specially.   We use the fancy verbatim mode and surround
identifiers with ! as in \verb'!identifier_with_underscore!' .

Subscripts written without braces are interpreted as part of the variable
or identifier name.

Subscripts written with braces are interpreted in CNL as function applications.

Thus, \verb!x_1! is a variable and  \verb!x_{1}!  is the function $x$ applied to $1$.
This works with multiple indices $a_{i j}$ (with a space between the $i$ and the $j$,
is translated to CNL \verb!a \sb (i j)!  The function application gets suppressed between
the $i$ and the $j$, so that this is equivalent to \verb!a i j!.   To activate function application
in the subscript, wrap the subscripts in an extra layer of parentheses \verb!a_{(f x)}! or
\verb!a_{\parenI{f x}}!.  The \verb!\parenI! control sequence is invisible in the pdf display,
but expands to parentheses in the CNL.

\subsubsection{labels}

Labels are converted to valid CNL identifiers by replacing
whitespace and non-alphanumeric characters to underscores.

\subsection{Special characters}

\subsubsection{superscripts}

\begin{itemize}
\item The use of the superscript mark \verb!^! should be restricted
to places where it acts as an infix (right associative) binary
operator for raising the term to a power.
\item When the superscript mark is used in connection with
subscripts as function call, the interpretation of \verb!f_i^n! is
the $n$th power of $f_i$, that is, $(f_i)^n$. 
Thus, this should be written in this
order rather than \verb!f^n_i! (which, would suggest the $n$th iterate
of $f$ applied to $i$).
\item the superscript character is treated like every other math symbol.
\item macros should be used when the meaning of a superscript is
not a power.
\end{itemize}

\subsubsection{invisible characters}

Invisible characters are generally given by macros
(with convention that their names end with $I$).  For example
\verb!\mulI! is the invisible multiplication operator.
\verb!x\mulI y! has pdf display $x y$, but the CNL image
preserves the binary operation \verb!x\mulI y!.

\subsection{control sequences}

\verb!\CNLcustom[k]\controlseq{pattern text}!   When translating \TeX\ to CNL use the pattern text to translate the control sequence \verb!\controlseq!
The integer $k$, which is optional, gives the number of braced arguments to read for expansion in the pattern text, after \verb!\controlseq!

For example, 
\verb!\CNLcustom[1]\parenI{ (#1) }! states that \verb!\parenI{X}! in the \TeX\ should translated to $(X)$ in the CNL.

For example,
\verb!\CNLcustom\oplus{ \vplus }!

A \verb!\noexpand! in the pattern text prevents the macro from being expanded, but the \verb!\noexpand! is not transcribed into the CNL image.

\verb!\CNLcustom\X{\noexpand\X}!.  Here \verb!\X! in the source becomes \verb!\X! in the image.

\verb!\CNLnoexpand[k]\controlseq! is equivalent to \verb!\CNLcustom[k]\controlseq{\noexpand\controlseq}!

\verb!\CNLdelete[k]\controlseq!  When translating to CNL, delete the control sequence \verb!\controlseq! together with $k$ curly braced arguments.
This is equivalent to giving empty expansion text \verb!\CNLcustom[k]\controlseq{}! 

\verb!\CNLexpand[k]\controlseq! expands the macro with $k$ arguments, but using the macro definition rather than \verb!\CNLcustom! pattern text.


\subsubsection{lists and curried functions}

Unlike in the source, in the CNL functions are generally curried.
In the CNL, lists are generally demarcated with square brackets and semicolons.

\verb!\list{(x1,...,xN)}! (which displays as $(x1,\ldots,xN)$)
becomes a list in the CNL image \verb![x1;...;xN]!  (Here the $x_i$s must have the same type.

\verb!\app{f}{(x1,....,xN)}! (which displays as $f(x1,...,xN)$) becomes curried in CNL \verb!((f) (x1) (x2) ... (xN))!

\subsubsection{stripped characters}

In translating to the CNL image, 
many formatting characters are removed from the \TeX\ source files.

This include instructions to change among different
\TeX\ modes:  

\verb!$ \( \) \begin{math} \end{math}!, 

\verb!\[ \] $$ \begin{displaymath} \end{displaymath}!.

Rules, glues, and space \verb!\  \, \medskip, \quad!, etc.

\subsubsection{curly braces}

Some curly braces are stripped.  Some are converted to
parentheses.  Here is the rule.  Curly braces marking a
subscript are replaced by parentheses.  Curly braces used
to demarcate control sequence arguments are stripped.



\subsubsection{ambiguous notations}

Lean elaboration does substantial work to resolve
notational ambiguities.  Some of these ambiguities
may be resolved directly by the author in the
\TeX\ source files, with notations that appear identical
in the pdf display.

For example,
\begin{itemize}
\item \verb!\closedInterval{a}{b}! -\qquad $[a,b]$
\item \verb!\LieCommutator{a}{b}! -\qquad $[a,b]$
\item \verb!\groupCommutator{a}{b}! -\qquad $[a,b]$
\item \verb!\image{f}{S}! -\qquad $f(S)$
\item \verb!f(S)! -\qquad $f(S)$
\item \verb!\fieldDegree{K}{F}! -\qquad $[K:F]$
\item \verb!\subgroupIndex{G}{H}! -\qquad $[G:H]$
\end{itemize}

\section{environments}

In CNL mode, everything is ignored outside the
\verb!\begin{cnl}...\end{cnl}! environment.

In CNL mode, there is a list of ignored environments, such as the
\verb!remark! environment.

In CNL mode, there is a list of translated environments, with custom
translation for each of them, such as the \verb!definition!  and
\verb!theorem! environments.


\subsection{specific environments}

Here we describe what happens to the \TeX sources, 
upon translation to CNL-image.

% https://latex.wikia.org/wiki/List_of_LaTeX_environments
% http://www.personal.ceu.hu/tex/environ.htm

\subsubsection{definition environment}

Theorems, Definitions, etc. can carry labels.  In fact, labels are
encouraged.  The labels should be valid atomic identifiers.  In the
CNL image, these labels become CNL labels.

Theorems, etc. can also carry optional names

\verb!\begin{theorem}[optional_name]\label{optional_name}!

When a name is given, the label should also appear, and the two should
be the same.


\subsubsection{paragraph environments}

\begin{description}
\item [center] Environment begin/end stripped, leaving enclosed text.
\item [flushleft, flushright, minipage, quotation, quote, verse]  begin/end as well as formatting options are stripped, leaving enclosed text.
\item [figure, picture, remark, thebibliography, titlepage] Environment begin/end and enclosed text stripped.
\item [tabbing]  begin/end stripped as well as tabbing characters \verb!\= \> \+ \-!
\item [tabular] not supported.
\end{description}

\subsubsection{equation alignment environments}

\begin{description}
\item [array] The begin/end and column format
instructions are stripped.  The array is converted
to a matrix (list of lists) in the CNL image.
\item [align, eqnarray, gather] begin/end and \verb!&! stripped.
A list is created with separator \verb!\\! translated
to separator \verb!;!
\item [equation,equation*]
begin/end and \verb!&! stripped.
Translation to CNL-Equation.  Any label is translated.
\item [multline, split]  begin/end and \verb!& \\! stripped.
\item [flalign, flalign*]  Not supported.
\end{description}

\subsubsection{matrix environments}

\begin{description}
\item [matrix, pmatrix, bmatrix, Bmatrix, vmatrix, Vmatrix, smallmatrix]   begin/end is stripped. The data is converted to a list of lists.  The list of lists is
given as an argument to a control sequence with the same name as the environment \verb!\pmatrix{[[0;1];[2;3]]}!
\item [cases] Not supported.
\end{description}



\subsection{itemize environment}

This includes itemize, enumerate, description.

Many objects can be itemized in the CNL.  We create
custom itemize environments for them.
This includes special \TeX\ support for match, function matching,
cases, inductive
and mutual inductive type declarations, structures, etc.
These custom itemize environments are translated to the
appropriate language constructs in the CNL image.


\section{outstanding issues}


\subsubsection{foreign accents}

In material originating in \TeX horizontal mode, 
the CNL image is
stripped of many of the foreign accents.

\begin{itemize}
\item G\"odel - Godel
\item \'etale - etale
\end{itemize}

\subsubsection{curly braces and font selection}



We generally wish to strip font variations \verb!{\it such as italics}! in
text mode.  Note that this doesn't fit the usual pattern of control sequence applied
to arguments.

\subsubsection{font selection in math}

Generally, we want to view two variables in different math fonts as
different variables.  We should generalize the definition of a variable in the CNL
to include constructs such as \verb!\mathcal{C}!  or $\mathcal{C}$, which would be
a separate variable from $C$.



%https://tex.stackexchange.com/questions/58098/what-are-all-the-font-styles-i-can-use-in-math-mode

\subsubsection{control sequences as variables}

We wish for it to be possible to use some control sequences
as variables: \verb!\alpha!, etc.



\subsubsection{ellipsis}


\end{document}
