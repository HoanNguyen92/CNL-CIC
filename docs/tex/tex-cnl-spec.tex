\documentclass[12pt]{amsart}


% PACKAGES
\usepackage{url}
\usepackage{amsmath}
\usepackage{amsthm}
\usepackage{amssymb}

% for underscores https://texfaq.org/FAQ-underscore
\usepackage{lmodern}
\usepackage[T1]{fontenc}

\usepackage{textcomp}


\usepackage[
bookmarksopen,
bookmarksdepth=2,
%breaklinks=true
colorlinks=true,
urlcolor=blue]{hyperref}

% GLOBAL FORMATTING
%\linenumbers
\parindent=0pt
\parskip=0.5\baselineskip
\raggedbottom

% TITLE AUTHOR DATE
\title{Notes on translating \TeX\ to a controlled natural language}

\date{July 17, 2019}                                           % Activate to display a given date or no date
\author{Thomas Hales}

% THEOREMS 
\newtheorem{definition}{Definition}
\newtheorem{theorem}[definition]{Theorem}
\newtheorem{lemma}[definition]{Lemma}
\newtheorem{specification}[definition]{Specification}

% COMMANDS
\renewcommand\labelitemi{-}


% ENVIRONMENTS

% \leavevmode\par is to make remark work when it is the first item in a subsection.
\newenvironment{remark}
{\leavevmode\par\begin{tabular}{|p{13cm}}\parskip=\baselineskip{\bf Remark.}}
{\end{tabular}}

\newenvironment{oblongo}{}{}

\newenvironment{prule}%
               {\begin{itemize}}%
               {\end{itemize}}
\newcommand{\ptem}{\item}
\newcommand{\nt}[1]{{\tt #1}}
\newcommand{\rw}{$\quad\to\quad$}



% DOCUMENT

\begin{document}
\maketitle



\section{\TeX-CNL translation specification}

The notes here add detail to the
 \href{https://github.com/formalabstracts/CNL-CIC/wiki/Conversion-from-LaTeX}{CNL-CIC Wiki}.

A \TeX\ file written in CNL-enabled \TeX\ can be processed 
by the \TeX\ engine in
two different modes: presentation mode and CNL mode.

A boolean flag {\tt isPresentation} in the \TeX\ source determines
which mode is used for processing.

The presentation mode is used to process the file as an ordinary
\TeX\ source, typeset in the usual way to a .dvi or .pdf file.  When
we talk about the pdf display, we mean the output with the
presentation mode.

The CNL mode is used to generate tables and indexes needed to
translate the \TeX\ source file into the CNL grammar.  The CNL image
means the output of the \TeX\ to CNL translation.  The CNL image is a
text file written in the CNL grammar.


\subsection{Differences in CNL-enabled files}

There are some visible differences between a typical \TeX\ source
and a CNL-enabled one.

The vertical bar | is redefined have an escape catcode (and \verb!\!
also remains an escape character).  The two escape characters are used
in different ways.  

Roughly, the vertical bar is used with macros that
have special translation rules to CNL.  In CNL mode, various control
sequences are redefined to include write commands to an external
file.  Macros are expanded by the \TeX\ engine and their expansions
are printed to the external file.  The \TeX\ to CNL translation software
uses this external file to perform the translation.  The two escape characters
are used to control where the write commands are inserted and whether
a given macro is expanded or left unexpanded in the dump to the external
file.  


In presentation mode, the vertical bar is the only symbol given a new
catcode.  Likewise CNL mode, the vertical bar is given a new catcode.
Moreover, the subscript and superscript symbols are given the other catcode.

\subsection{Variables and Identifiers}

Identifiers can appear with the underscore character.  They must be
processed specially.

\subsubsection{identifiers}
In the \TeX\ source, 
\begin{itemize}
\item write \verb!|var|x|! or (\verb!$x$!) 
to produce a variable $x$.
\item Write \verb!|id|my_identifier|! to
produce the identifier \verb!my_identifier!.
\item Write \verb!|id|mynamespace.my_namespace|! for
a hierarchical identifier.
\item Write \verb!|hier|mynamespace|my_namespace|! for
a hierarchical identifier in which \verb!mynamespace! is
not displayed in the pdf, but appears in the CNL.
\item Insert curly braces after an underscore to make parts
of the identifier display in the pdf as an underscore.
In the CNL image, the braces are stripped, and the content
between the braces becomes part of the identifier.
\item To make the subscript a function call, place it outside
the vertical bars: \verb!|id|A|_{ij}! for $\hbox{|id|A|}_{ij}$.
\item In CNL-mode, the parentheses are around the function
and the subscript, not around terms within the subscript.
\verb!f_{i j}! is translated to $(f~i~j)$ rather than $f (i j)$
\item In CNL-mode, the underscore is given the catcode other.
\end{itemize}

\subsubsection{labels}

Labels must be valid CNL identifiers.


References \verb!\ref{xyz}! or \verb!\eqref{xyz}!  in the
\TeX\ translate to the corresponding labels statements {\it by label}
in CNL.


\subsection{Special characters}

\subsubsection{superscripts}

\begin{itemize}
\item The use of the superscript mark \verb!^! should be restricted
to places where it acts as an infix (right associative) binary
operator for raising the term to a power.
\item When the superscript mark is used in connection with
subscripts as function call, the interpretation of \verb!f_i^n! is
the $n$th power of $f_i$, that is, $(f_i)^n$. 
Thus, this should be written in this
order rather than \verb!f^n_i! (which, would suggest the $n$th iterate
of $f$ applied to $i$).
\item the superscript mark is given the other catcode in CNL-mode.
\item macros should be used when the meaning of a superscript is
not a power.
\end{itemize}

\subsubsection{invisible characters}

Invisible characters are generally given by macros
(with convention that their names end with $I$).  For example
\verb!\mulI! is the invisible multiplication operator.
\verb!x\mulI y! has pdf display $x y$, but the CNL image
preserves the binary operation \verb!x\mulI y!.



\subsubsection{stripped characters}

In translating to the CNL image, 
many formatting characters are removed from the \TeX\ source files.

This include instructions to change among different
\TeX\ modes:  

\verb!$ \( \) \begin{math} \end{math}!, 

\verb!\[ \] $$ \begin{displaymath} \end{displaymath}!.

Rules, glues, and space \verb!\  \, \medskip, \quad!, etc.

Most curly braces are stripped.  Some are converted to
parentheses.


In material originating in \TeX horizontal mode, 
the CNL image is
stripped of many of the foreign accents.

\begin{itemize}
\item G\"odel - Godel
\item \'etale - etale
\end{itemize}




\subsubsection{special characters}

It is better to use the letter version of control
sequences rather than the escaped forms:
use \verb!\percent! not \verb!\%!, \verb!\dollar! not
\verb!\$! etc.

\subsubsection{ambiguous notations}

Lean elaboration does substantial work to resolve
notational ambiguities.  Some of these ambiguities
may be resolved directly by the author in the
\TeX\ source files, with notations that appear identical
in the pdf display.

For example,
\begin{itemize}
\item \verb!\closedInterval{a}{b}! -\qquad $[a,b]$
\item \verb!\LieCommutator{a}{b}! -\qquad $[a,b]$
\item \verb!\groupCommutator{a}{b}! -\qquad $[a,b]$
\item \verb!\image{f}{S}! -\qquad $f(S)$
\item \verb!f(S)! -\qquad $f(S)$
\item \verb!\fieldDegree{K}{F}! -\qquad $[K:F]$
\item \verb!\subgroupIndex{G}{H}! -\qquad $[G:H]$
\end{itemize}

\section{environments}

In CNL mode, everything is ignored outside the
\verb!\begin{cnl}...\end{cnl}! environment.

In CNL mode, there is a list of ignored environments, such as the
\verb!remark! environment.

In CNL mode, there is a list of translated environments, with custom
translation for each of them, such as the \verb!definition!  and
\verb!theorem! environments.


\subsection{specific environments}

Here we describe what happens to the \TeX sources, 
upon translation to CNL-image.

% https://latex.wikia.org/wiki/List_of_LaTeX_environments
% http://www.personal.ceu.hu/tex/environ.htm

\subsubsection{definition environment}

Theorems, Definitions, etc. can carry labels.  In fact, labels are
encouraged.  The labels should be valid atomic identifiers.  In the
CNL image, these labels become CNL labels.

Theorems, etc. can also carry optional names

\verb!\begin{theorem}[optional_name]\label{optional_name}!

When a name is given, the label should also appear, and the two should
be the same.


\subsubsection{paragraph environments}

\begin{description}
\item [center] Environment begin/end stripped, leaving enclosed text.
\item [flushleft, flushright, minipage, quotation, quote, verse]  begin/end as well as formatting options are stripped, leaving enclosed text.
\item [figure, picture, remark, thebibliography, titlepage] Environment begin/end and enclosed text stripped.
\item [tabbing]  begin/end stripped as well as tabbing characters \verb!\= \> \+ \-!
\item [tabular] not supported.
\end{description}

\subsubsection{equation alignment environments}

\begin{description}
\item [array] The begin/end and column format
instructions are stripped.  The array is converted
to a matrix (list of lists) in the CNL image.
\item [align, eqnarray, gather] begin/end and \verb!&! stripped.
A list is created with separator \verb!\\! translated
to separator \verb!;!
\item [equation,equation*]
begin/end and \verb!&! stripped.
Translation to CNL-Equation.  Any label is translated.
\item [multline, split]  begin/end and \verb!& \\! stripped.
\item [flalign, flalign*]  Not supported.
\end{description}

\subsubsection{matrix environments}

\begin{description}
\item [matrix, pmatrix, bmatrix, Bmatrix, vmatrix, Vmatrix, smallmatrix]   begin/end is stripped. The data is converted to a list of lists.
\item [cases] Not supported.
\end{description}



\subsection{itemize environment}

This includes itemize, enumerate, description.

Many objects can be itemized in the CNL.  We create
custom itemize environments for them.
This includes special \TeX\ support for match, function matching,
cases, inductive
and mutual inductive type declarations, structures, etc.
These custom itemize environments are translated to the
appropriate language constructs in the CNL image.


\end{document}
