Colada is a \emph{controlled natural language} for writing mathematics.

This document is a specification and user manual for the Colada controlled natural language.

\section{Introduction}

The practice of mathematics is often described in ideal terms as a
formal deductive system.  A precise language is specified.  The axioms
of mathematics are written down in this language.  The permissible
rules of logic are spelled out in complete detail. New definitions are
built up systematically from previous definitions.  Theorems are
proved in a series of deductions grounded in the axioms and rules of
logic.  A mathematical proof that is carried out according to this
ideal is called a formal proof.  


There is a long history behind this style of doing mathematics, going
back to early Greek mathematics (Thales, Pythagoreas, Euclid, \ldots).
Various foundational systems (such as Zermelo Fraenkel Choice set
theory or type theory) are adequate for large parts of mathematics as
currently practiced.

Progress over the centuries has culminated in software proof
assistants.  These are software systems that verify each step of a
formal proof.  Major proof assistants include Coq, Lean, HOL (in
various dialects), and Mizar.  Mathematicians are generally aware
of software proof assistants, but only rarely does a mathematician
learn to formalize proofs in one of these
systems.  Most users are still computer scientists and their students.


The usual means of written communication of mathematicians depart
significantly from this ideal and the vast majority of mathematical
theorems are written in English (or some other natural langauge),
typed into \TeX, and then typeset into pdf documents.  Sometimes mathematical research
is supplemented with calculations in computer algebra systems or
in other computational packages.

The aim of this project is to narrow the gap between the proof
assistant and current mathematical written communication.


By a controlled natural language for mathematics (CNL), we mean
an artificial language for the communication of mathematics that
\begin{itemize}
  \item is deliberately designed with precise computer-readable syntax
    and semantics,
    \item is based on a single natural language (which
      for us will be English),
      \item and is broadly understood at least
in an intuitive way by mathematically literate speakers of
the natural language.
\end{itemize}




Some of design goals for this language have been
\begin{itemize}
\item Writing documents in this language
  should be similar to writing \LaTeX.
\item Reading documents in this language should be similar to reading
  ordinary mathematical texts typeset in \LaTeX.  A short explanation
  should be all that is needed to explain the special conventions of
  the language.
\item The logical and mathematical foundations of the system are
  provided by the Calculus of Inductive Constructions (CiC) as
  implemented in the Lean theorem prover.  Documents written in the
  language should compile to CiC.
\end{itemize}









\section{Lexical Structure}

The character set for Colada is ASCII.  However, future
versions of the language will use unicode (UTF8) characters.

\subsection{Character Categories}

Characters are divided into the following categories.

Comment: \% 

Whitespace: (space) (tab) (linefeed) (carriagereturn).

Alphabet: a-Z, A-Z.

Digit: 0-9.

Alphanumeric: alphabet, digit, and ' \_

Punctuation: . ; ,

Delimiter: ( ) [ ] { }

ControlCharacter $\backslash$

Symbol:  All other ASCII characters.


\section{Control Sequences}

\section{}

\section{Patterns}


\section{The Calculus of Inductive Constructions}


\section{Converting \TeX\ to Colada}

\section{preprocessing \TeX}

\section{Sample Texts}

