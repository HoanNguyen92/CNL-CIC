\documentclass[12pt]{article}


% PACKAGES
\usepackage{url}
\usepackage{amsmath}
\usepackage{amsthm}
\usepackage{amssymb}
\usepackage{xcolor}
\usepackage{currfile}
\usepackage{fancyvrb}
\usepackage{xparse} % needed for \ellipsis control sequence in cnl-style
\usepackage{enumitem} % for topsep=0pt
\usepackage{colonequals}

% for underscores https://texfaq.org/FAQ-underscore
\usepackage{lmodern}
\usepackage[T1]{fontenc}
\usepackage{textcomp}
\usepackage{lineno}

\usepackage[
bookmarksopen,
bookmarksdepth=2,
%breaklinks=true
colorlinks=true,
urlcolor=blue]{hyperref}

% GLOBAL FORMATTING
%\linenumbers
\parindent=0pt
\parskip=0.5\baselineskip
\raggedbottom

% TITLE AUTHOR DATE
\title{Reader's Guide to the Colada Language}

\date{September 18, 2019}
\author{Thomas Hales}

% THEOREMS
\newtheorem{definition}{Definition}
\newtheorem{theorem}[definition]{Theorem}
\newtheorem{lemma}[definition]{Lemma}
\numberwithin{definition}{section}
%\newtheorem{specification}[definition]{Specification}


% DOCUMENT

\begin{document}
\maketitle

\setcounter{tocdepth}{1}
\tableofcontents
\newpage

\newcommand{\Nat}{{\mathbb N}}
\newcommand{\Int}{{\mathbb Z}}
\newcommand{\Real}{{\mathbb R}}

This is a reader's guide to the Colada language, as in Pina Colada.  The name
stands for {\bf CO}ntrolled {\bf LA}nguage {\bf DA}ta.

\section{Introduction}


By a controlled natural language for mathematics (CNL), we mean an
artificial language for the communication of mathematics that
\begin{itemize}
\item is deliberately designed with precise computer-readable syntax
  and semantics,
\item is based on a single natural language (which for us will be
  English),
\item and is broadly understood at least in an intuitive way by
  mathematically literate speakers of the natural language.
\end{itemize}




Here are some design goals.
\begin{itemize}
\item The language should narrow the gap between formalization and
  current mathematical written communication.
\item Writing documents in this language should be similar to writing
  \LaTeX.
\item Reading documents in this language should be similar to reading
  ordinary mathematical texts typeset in \LaTeX.  A short explanation
  should be all that is needed to explain the special conventions of
  the language.
\item The logical and mathematical foundations of the system are
  provided by the Calculus of Inductive Constructions (CiC) as
  implemented in the Lean theorem prover.  Documents written in the
  language should compile to CiC.
\end{itemize}


This is a short guide to reading texts that have been written in the
Colada controlled natural language.  My goal was to keep this reader's
guide really short.  It is intended to provide \emph{quick start} that
avoids the technicalities.  The intended reader of this guide is a
with no experience in reading Colada documents and no experience in type theory.

We have tried to design a language that is rather close to current
conventions of written mathematical communication.  However, an exact
fit is not possible for various reasons.

\begin{itemize}
\item Most mathematical writing is imprecise.
\item The semantics of our language is dependent type theory.
\end{itemize}

It is easier to read a Colada document than to write one.%
\footnote{Just as it is easier to watch a film than to direct one, to
  use Stephen Watt's apt analogy.}  
%
This guide is meant to help mathematicians read the language on an
intuitive level, with an emphasis on those aspects of the language
that are most unlike traditional mathematical writing.

A longer separate document (the writer's guide to the Colada language)
in preparation will give a careful specification of the language and a
tutorial on writing Colada texts.

As a first approximation, the Colada language can be viewed as the
disjoint union of the syntaxes of \LaTeX, dependent type theory, and
Forthel, glued together along a common semantic core.%
%
\footnote{The intended target semantics of the Colada language is the
  calculus of inductive constructions, as implemented in the Lean
  theorem prover.}
%
From a slightly simpler perspective, Colada documents are just
\LaTeX\ documents with precisely described semantics.  Colada documents are
generally written in \LaTeX, and this reader's guide describes how to
read the pdf document, as typeset with \LaTeX.

The language has been designed as a language in which mathematical
definitions and theorem statements can be expressed in a precise,
yet readable manner.   Stating something precisely in a computer
readable format is no guarantee of correctness.  The reader be warned!

The language includes syntax for mathematical
proofs, but that has not been our primary aim.  There is no guarantee
that a proof written in in our language can be reconstructed as a formal
proof in a proof assistant. The reader be warned!

\section{Type theory notation}

The biggest adjustment in reading Colada texts is that the semantics
are based on type theory rather than set theory.

Set theory and type theory are two different ways to organize much of
mathematics into a comprehensive system.  Type theory is rather like
set theory, but with a few important differences.

The Venn diagram is a universally recognized picture of sets. Venn
diagrams depict sets as collections of objects that intersect in
various ways. Types can also be depicted as collections of objects.
However, types are always disjoint from one another (and this is the
absolutely crucial difference from sets).  Thus, a picture of type
theory looks more like that non-overlapping bricks on a wall, where
each type is represented as a brick.

The primary difficulty of trying to use set theory as a foundational
system on a computer is that sets freely mix (in Venn diagram style).
On a computer, set theory becomes like a distasteful scavenger soup in which
everything is thrown into the same pot.

A disciplined type system organizes data on a computer.  The primary
difficulty in trying to use type theory as a foundational system for
mathematics is that types to not mix at all (following the metaphor of
backed bricks on a wall).  The challenge is to make the types mix a
just a bit.  For example, we do not want the natural number zero and
the real number zero (which are distinct objects in type theory) to be
completely divorced from each other.

In type theory, every object has a unique type.%
%
\footnote{Throughout these notes, we refer to a particular version of
  type theory called the calculus of inductive constructions with
  non-cumulative universes, as implemented in the Lean theorem
  prover.}
%
This is the \emph{fundamental theorem} of type theory.



\subsection{Colons are everywhere}

We turn to a description of the syntax.

\[
\boxed{(n : \Nat)}
\]
means that $n$ has type $\Nat$.  In set theory, we would write that
$n$ is a member of the set $\Nat$, or $n \in \Nat$.  To a rough
approximation, the reader can read $(n : \Nat)$ as $n\in \Nat$.
By definition, $0$ is the smallest natural number.

\[
\boxed{x\in X}
\]

The notation $x\in X$ still appears in our language, but it is not a
primitive of our language the way it is in set theory.  An author of a
document is free to extend the meaning of $\in$ to new contexts.  For
example, an author might define $x\in X$ to mean that a natural number $x$
is stored in a leaf of a tree structure $X$.

\[
\boxed{(f : \Real \to \Real)}
\]
means that $f$ has type $\Real\to\Real$.  An object $f$ of type
$\Real\to\Real$ is a function from domain $\Real$ to codomain $\Real$
(as a naive set-theoretic reading of the boxed expression might
suggest).  That is, $\Real\to\Real$ denotes the function space, and
$f$ is a member of that function space.  In brief, there
is little harm in reading $f :\Real\to\Real$ just like in set theory,
even if a type theorist reads it in a slightly different way.

\[
\boxed{(f : \Nat \to\Int \to \Real)}\quad \text{or equivalently}\quad \boxed{(f:\Nat\to(\Int\to\Real))}
\]
means that $f$ is a function with domain $\Nat$ and codomain the
function space $\Int\to\Real$.  Thus, we have a function
$f(n):\Int\to\Real$ for each $n\in \Nat$, and $f(n)(r):\Real$ for each
$r\in\Int$.  Note that the function space $\Nat\to\Int\to\Real$ is in
bijective correspondence $f\leftrightarrow F$ with the function space
$(\Nat\times \Int)\to\Real$, where $f(n)(r) = F(n,r)$. This
corresponds with the natural bijection between $(\Real^{\Int})^{\Nat}$
and $\Real^{\Nat\times\Int}$ in set theory, where $Y^X$ denotes the space
of functions from $X$ to $Y$.


It is very common in type theory to construct functions whose type is
a chain of arrows (for example, $X_1\to (X_2\to (X_3 \to X_4))$, or
more concisely, $X_1\to X_2\to X_3 \to X_4$).  Such functions $f$ are
said to be curried (named for the logician Curry), which are generally
preferred to their uncurried $F:X_1\times X_2\times X_3\to X_4$
counterparts.

\[
\boxed{(p : 1 < 2)}
\]
A mathematical proposition (such as $1<2$) can appear to the right
of a colon.  In this case, the boxed expressed should be read
``$p$ is a proof of the proposition $1<2$''.  

When $X$ and $Y$ are
propositions, $p: X\to Y$ can be read both as ``$p$ is a proof
of the proposition `$X$ implies $Y$' ''  and as ``$p$ is a function
that transforms a proof of $X$ into a proof of $Y$.''

\section{Functions}

Here is the notation for functions.


\[
\boxed{f\ x}
\]
means $f(x)$.  Generally the parentheses are not written, when
applying a function $f$ to its argument.

\[
\boxed{f\ x\ y}
\]
means $f(x)(y)$, where $f$ is a curried function $f:X\to Y\to Z$.


\begin{center}
\fbox{\begin{minipage}{6em}
$\displaystyle\begin{aligned}[topsep=1pt]
f:X &\to Y \\
x &\mapsto f\ x
\end{aligned}$
\end{minipage}}
\end{center}
gives a function $f$ of type $X\to Y$, specified by the rule $x\mapsto f\ x$.

\begin{center}
\fbox{\begin{minipage}{11em}
$\displaystyle\begin{aligned}[topsep=1pt]
\Int\to (n:\Int) &\to \Int/n\Int \\
a\quad n &\mapsto a \mod n
\end{aligned}$
\end{minipage}}
\end{center}
This is a \emph{dependent function}, in which the codomain depends on
the second of two integer arguments to the function.  Pay careful
attention to the syntax%
\footnote{Type theorist call the type of a dependent function a
  $\Pi$-type, by analogy with the set theoretic product $\Pi_{n\in\Nat}
  (\Int/n\Int)$, the set of sequences $(b_n)_{n\in\Nat}$ such that
  $b_n\in \Int/n\Int$.}
%
of the type: $\Int\to (n:\Int) \to
\Int/n\Int$. The second (curried) argument of the function is an
integer, which is labeled as $n$ in the type signature in order to
specify the codomain.

Dependent functions are one of the most striking features of our type
theory.  Dependent functions appear more frequently here in set
theory, because in set theory a multiplicity of codomains can be
replaced with their union, to reduce to a single codomain.%
\footnote{In type theory a heterogeneous union is not so
  straightforward.}

\[
\boxed{\text{\bf fun}\ x\ := y}
\]
means the same as $x\mapsto y$.  The keyword {\bf fun} introduces the
rule.%
\footnote{{\bf fun} (or $\mapsto$) is our preferred notation for a
  $\lambda$-term in the $\lambda$-calculus.  We avoid the Greek letter
  $\lambda$ in order to free up the letter for the many other uses of
  $\lambda$ in mathematics (and also to remain closer to the preferred
  convention $\mapsto$ of mathematicians).  The variable $a$ is a
  bound variable.}  In general, $x := y$ means that the left-hand side
$x$ is defined to be $y$.

\[
\boxed{(\text{\bf fun}\ x\ y\ := x + y)\ 1\ 2 = 1+2}\quad\text{or}\quad
\boxed{(x\ y\ \mapsto x+y)\ 1\ 2 = 1+2}
\]
is an anonymous 
function, applied to curried arguments $1$ and $2$, yielding $1+2$.



\begin{center}
\fbox{\begin{minipage}{8em}
\text{\bf function}\par
\begin{tabular}{@{\quad\normalfont| }lll}
$0$ &$:=$ &$0$\\
$k+1$ &$:=$ &$k$
\end{tabular}
\end{minipage}
}
\end{center}
This function is the truncated decrement on the natural numbers, which maps $0$ to $0$
and $k+1$ to $k$.   The different cases are marked with a vertical stroke $|$.
We use {\bf function} when the function definition has different cases
and use {\bf fun} when the function has multiple curried arguments.


\section{Formulas}

We try to make formulas appear similar to those appearing
in conventional mathematical texts.

New notations may be introduced as a document progresses.

\[
\boxed{1 + 1 + 1}
\]
means $(1+1)+1$ because infix\footnote{By an infix notation,
we mean a function (such as $+$) that is inserted between its
arguments.  A postfix notation (such as $3!$) appears after
its argument. A prefix notation (such as $\sin\ \pi$) appears before
its argument.} addition is left associative.  Binary
operations can be left associative or right associative, and its
associativity determines how to reinsert parentheses into chains of
the operation.

\[
\boxed{1 + 2 * 3}
\]
means $1 + (2*3)$ because 
multiplication has higher precedence than
addition.  The precedence of each binary operation is an integer.  The
precedence of every non-logical operation (such as addition,
multiplication, subtraction, and powers) is positive. The precedence
of predicates (such as $=$, $<$, $\subset$, $\in$) is zero. The
precedence of every logical operation (such as and, or, iff) is
negative.  New operations are assigned a precedence when introduced
into a document.

\[
\boxed{u < x = y = z \le w}
\]
means $u < x$ and $x = y$ and $y = z$ and $z\le w$.  Predicates all have the
same precedence level and they may be chained as in this example.
The chained predicates mean the same as if broken into separate predicates
joined by \emph{and}.

\[
\boxed{\cos \pi/2 + 4!}
\]
means $(\cos\pi)/2 + (4!)$.  Function applications (here $\cos$
applied to $x$) and postfix notations (here the factorial applied to
$4$) always have higher precedence than all infix binary operations
(here $+$ and $/$).


We try to retain standard mathematical notation wherever possible for
numbers, polynomials, matrices, limits, derivatives, integrals,
ellipsis $\ldots$.  However, there are some noteworthy distinctions,
where we need to be pedantic to get everything just right.
\[
\boxed{x^7 - 7 x + 3}
\]
is a polynomial of degree $7$ in $x$. However, the internal representation
of polynomials is by a function $0\mapsto3,1\mapsto-7,7\mapsto1$
of finite support, which maps exponents to coefficients,
in which the $x$ has disappeared.

\[
\boxed{\frac{d (x\mapsto x^3)}{dx} = (x \mapsto 3 x^2)}\quad\text{versus}\quad
\boxed{\frac{d\,x^3}{dx} = 3 x^2}
\]
Derivative notation $dx^3/dx$ is problematic. We should differentiate
a function $x \mapsto x^3$ rather than a term $x^3$.  However, the
variable $x$ in the function is bound, and it cannot legitimately%
%
\footnote{Our language does not support \emph{bound variable
    anaphors}, as they are called. In fact, our language has no
  pronouns at all. Phrases such as \emph{``we have''} in a Colada text
  are empty filler words that are ignored by the semantics.}  
%
occur outside the function (in $dx$).  We might choose to provide
\emph{syntactic sugar} which allows the illegitimate syntax to
continue to be used, with a translation behind the scenes into
something that makes sense.  The reader is hereby warned that sugary
syntax (especially when it condones questionable mathematical
practice) can be introduced by document authors, even when deceptive
and disturbing.




\section{Programming Notation}

The Colada language is written with a precise syntax that can 
be interpreted directly by computer.  Because of that, some of the
notations are similar to those found in programming languages.%
\footnote{Functional programming languages such as OCaml and Haskell
  share significant amounts of syntax with Colada.}

\[
\boxed{[1;2;3]} \quad\text{versus}\quad \boxed{(1,2,3)}
\]
A list $[1;2;3]$ is very much like a tuple $(1,2,3)$, but it
carries the additional requirement that all of its entries have the
same type (in this case the entries have type $\Nat$).  Lists are very
common.  There is a large library of functions (with special
notations) designed to make lists easy to work with.  For example $0
\coloncolon[1;2;3]$ attaches $0$ to the beginning of the list to
produce $[0;1;2;3]$; and $[1;2;3] @ [4;5;6]$ appends the second list
to the first to produce $[1;2;3;4;5;6]$.

Each component of an a tuple $(x,y,z)$ can have a different type.


match,  case, inductive types, structures.

\section{Enlish text}

The Colada language contains a initial vocabulary%
\footnote{This list continues to evolve as we develop the language.}
%
of words that function according to the rules of the grammar:

\begin{quote}
{\it a, all, an, analysis, and, any, are, article, as, associativity,
assume, axiom, be, by, called, can, canonical, case, choose,
classifier, classifiers, coercion, conjecture, contradiction,
contrary, corollary, def, define, defined, definition, denote, do,
document, does, dump, each, else, end, endsection, endsubdivision,
endsubsection, endsubsubsection, equal, equation, every, exhaustive,
exist, exists, exit, false, fix, for, forall, formula, fun, function,
has, have, having, hence, holding, hypothesis, if, iff, in, inferring,
indeed, induction, inductive, introduce, is, it, left, lemma, let,
library, make, map, match, no, not, notational, notation,
notationless, obvious, of, off, on, only, ontored, or, over, pairwise,
parameter, parameters, precedence, printgoal, proof, properties,
property, prove, proposition, propped, qed, quotient, read, record,
register, recursion, right, said, say, section, show, some, stand,
structure, subdivision, subsection, subsubsection, such, suppose,
synonym, synonyms, take, that, the, then, theorem, there, therefore,
thesis, this, timelimit, to, total, trivial, true, type, unique, us,
we, well, welldefined, where, with, wrong, yes.}
\end{quote}


The language also recognizes more than a hundred stock phrases that
can be used in a proof, including

\begin{quote}
{\it certainly, clearly, for example, furthermore, in particular, however,
hence, likewise, it is enough to show that, obviously, \ldots.}
\end{quote}
% phrase_lists.txt

The initial vocabulary and stock phrases can be augmented through
definitions and macros.  Every word written in the controlled natural
language is either in the initial vocubulary, a word in a stock
phrase, or introduced by macro or definition.





\end{document}

