\documentclass{easychair}


% PACKAGES
\usepackage{url}
\usepackage{amsmath}
\usepackage{amsthm}
\usepackage{amssymb}

% for underscores https://texfaq.org/FAQ-underscore
\usepackage{lmodern}
\usepackage[T1]{fontenc}
\usepackage{textcomp}
\usepackage{lineno}

%\usepackage[
%bookmarksopen,
%bookmarksdepth=2,
%%breaklinks=true
%colorlinks=true,
%urlcolor=blue]{hyperref}

% GLOBAL FORMATTING
%\linenumbers
\parindent=0pt
\parskip=0.5\baselineskip
\raggedbottom

% TITLE AUTHOR DATE
\title{A Controlled Natural Language for Type Theory%
\thanks{I thank Peter Koepke for introducing me to the field and Jesse Han for his
preliminary work on parsing.  
Research is supported by Sloan Foundation grant G-2018-10067.
Source code and examples are found at \url{github.com/formalabstracts/CNL-CIC}.}
}

%\date{November 17, 2019}
\author{Thomas Hales}

\institute{
University of Pittsburgh\\
Pittsburgh, PA, USA
}

\authorrunning{T.Hales}
\titlerunning{Controlled Natural Language}



% THEOREMS 
\newtheorem{definition}{Definition}
\newtheorem{theorem}[definition]{Theorem}
\newtheorem{lemma}[definition]{Lemma}
\newtheorem{specification}[definition]{Specification}

% COMMANDS
\renewcommand{\iff}{\leftrightarrow}
\newcommand{\Prop}{\text{\tt Prop}}
\newcommand{\Type}{\text{\tt Type}}
\newcommand{\fld}{\textasciicircum}
\newcommand{\dequiv}{\mathrel{:=}} %{\mathrel{:\equiv}}
\newcommand{\Nat}{\ensuremath{{\mathbb N}}}
\newcommand{\Real}{\ensuremath{{\mathbb R}}}
\newcommand{\df}[1]{\text{\bf #1}}
\newcommand{\h}[1]{\text{#1}}
\newcommand{\join}{\lor}
\newcommand{\Mid}{\mathrel{\|}}
\newcommand{\comment}[1]{\%- \nobreak{#1}}
\renewcommand{\~}{\ }
\newcommand{\ignore}[1]{}
%\newcommand{\remark}[1]{(#1)}
\renewcommand{\_}{\textunderscore}
\renewcommand\labelitemi{-}
\renewcommand{\qed}{\ensuremath{\square}}

% ENVIRONMENTS

% \leavevmode\par is to make remark work when it is the first item in a subsection.
\newenvironment{remark}
{\leavevmode\par\begin{tabular}{|p{13cm}}\parskip=\baselineskip{\bf Remark.}}
{\end{tabular}}

\newenvironment{oblongo}{}{}

\newenvironment{prule}%
               {\begin{itemize}}%
               {\end{itemize}}
\newcommand{\ptem}{\item}
\newcommand{\nt}[1]{{\tt #1}}
\newcommand{\rw}{$\quad\to\quad$}



% DOCUMENT

\begin{document}
\maketitle

\section{Introduction}

This abstract describes the design and development of a
controlled natural language for mathematics that has the Lean 
theorem prover as intended target.  We call this language Colada (short for
\emph{Co}ntrolled \emph{la}nguage \emph{da}ta).
Documents in our dialect are written in a specially prepared \LaTeX\ file.
Our aim is to capture definitions and theorem statements from the published
mathematical literature in our dialect, but 
checking mathematical proofs is beyond the scope of our project.



Our design grows out of previous controlled natural
languages for mathematics (specifically, Forthel-Naproche-SAD), as
described in Peter Koepke's AITP 2019 talk, which 
exhibited some short proofs written in fluent English that can be read and checked by their software. 

We use
Forthel as the generic name for any of
the dialects inspired by Forthel (a 
controlled natural language developed by Paskevich), including Colada. 
We refer to the Colada language as
\emph{our dialect}.
Our dialect differs from others in that our semantic target is the logical Calculus of Inductive
Constructions (CiC) as implemented in the Lean theorem prover, 
instead of first-order logic.  
Our dialect can be viewed as a fusion of three different syntactic traditions:
Forthel syntax, \LaTeX\ syntax, and Lean theorem-prover
syntax.  
From another perspective, our dialect might be viewed as a  
mountain of syntactic sugar for Lean.




\section{Controlled Natural Languages (CNL)}\label{sub:CNL}

By controlled natural language for mathematics (CNL), we mean an
artificial language for the communication of mathematics that is (1)
designed in a deliberate and explicit way with precise
computer-readable syntax and semantics, (2) based on a single natural
language (which for us is  English), and (3) broadly
understood at least in an intuitive way by mathematically literate
speakers of the natural language.

CNLs can achieve a much higher degree of English fluency than other
proof-checking languages.  


Following a divide-and-conquer strategy, our
basic aim is to develop a technology that lies roughly midway
between current practice of research mathematicians and the current
practice within the proof assistant community.  

\section{Research to Date}

Our specific research contributions to date are as follows.

We have a design and specification of a controlled natural
language.
Like other Forthel dialects, our grammar is not a context-free.
However, it is similar to a context-free grammar by being specified through
production rules on terminal and nonterminal symbols.
Users may extend the grammar with new mathematical notation
and constructs:  the language
contains syntax for the extension of its own syntax. 

The lexical structure of our dialect is specified in sedlex, a lexical
generator tool for OCaml.
Our dialect has been specified in menhir, an OCaml-based
parser-generator tool for LR(1) grammars.  (Although our dialect is
not an LR(1) grammar, which prevents menhir from automatically
generating a parser, the software checks that our grammar is well-formed.)

We believe that some complexity is justified (and even required) to
capture widespread mathematical idioms and formulas, the syntax of type
theory, and their
interactions. Our grammar is recursive to an extraordinary degree.
The grammar has about 350 nonterminals, about 350 production
rules, and about 150 context-dependent key words.   User syntax
extensions build on that base.

We keep most features of Forthel, such its handling of synonyms, noun
phrases, verbs, and adjectives; and its grammar extension mechanisms.
We have added many additional features such as plural formation for nouns and verbs,
operator precedence parsing (with
user-specified precedence levels and associativities); scoping of variables;
syntax for \LaTeX\ macros; and dependent type theory including
inductive and mutual inductive types, structures, and lambda
terms.

A parser for our grammar has been implemented in OCaml, building substantially on
the parser combinator library that John Harrison wrote to parse HOL
Light.   

Future work will transform parsed output to type-checked
terms in Lean.  Another future project is syntax highlighting and auto-completion tools 
for our
dialect in editors such as emacs and VSCode. We also plan to develop large
mathematical libraries in our dialect.

We have written software that takes a specially prepared
\LaTeX\ file as input and strips away the non-semantic content  
(such as headers, spaces and other layout, graphics,
remarks, and dollar signs) and
outputs raw CNL.
The key to beautifully typeset \TeX\ documents is a
dual expansion system for macros.  The \TeX\ engine expands macros in
the usual way, but the CNL engine expands some macros according to an
independent semantic specification.   

Many examples will be given during the AITP presentation to show that English
fluency is obtained without loss of semantic content.  We skip them here for lack of space.

We believe our language will find novel applications to search, document
analysis, and document transformation.  
Ultimately each word, symbol, and phrase of a math document written in our dialect can be 
accounted for. 

\bigskip
\noindent\rule{2cm}{0.4pt}


This work is part of the Formal Abstracts project supported by the Sloan Foundation, 
which aims to capture all the major
definitions and theorems of mathematics in a format that is both human and computer
friendly.


\end{document}
