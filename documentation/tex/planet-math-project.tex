\documentclass[12pt]{article}


% PACKAGES
\usepackage{url}
\usepackage{amsmath}
\usepackage{amsthm}
\usepackage{amssymb}
\usepackage{xcolor}
\usepackage{currfile}
\usepackage{fancyvrb}
\usepackage{xparse} % needed for \ellipsis control sequence in cnl-style
\usepackage{enumitem} % for topsep=0pt
\usepackage{colonequals}

% for underscores https://texfaq.org/FAQ-underscore
\usepackage{lmodern}
\usepackage[T1]{fontenc}
\usepackage{textcomp}
\usepackage{lineno}

\usepackage[
bookmarksopen,
bookmarksdepth=2,
%breaklinks=true
colorlinks=true,
urlcolor=blue]{hyperref}

% GLOBAL FORMATTING
%\linenumbers
\parindent=0pt
\parskip=0.5\baselineskip
\raggedbottom

% TITLE AUTHOR DATE
\title{Guidelines for the Planet-Math translation to Colada}

\date{October 30, 2019}
\author{Thomas Hales}

% THEOREMS
\newtheorem{definition}{Definition}
\newtheorem{theorem}[definition]{Theorem}
\newtheorem{lemma}[definition]{Lemma}
\numberwithin{definition}{section}
%\newtheorem{specification}[definition]{Specification}


% DOCUMENT

\begin{document}
\maketitle

\setcounter{tocdepth}{2}
\tableofcontents
\newpage

\newcommand{\Nat}{{\mathbb N}}
\newcommand{\Int}{{\mathbb Z}}
\newcommand{\Real}{{\mathbb R}}

This is a document describing the proposed project to translate
the entire body of PlanetMath (PM) into the Colada controlled
natural language.
A pdf document containing all PlanetMath would run
(an estimated) 
several thousand pages.


References are \href{https://planetmath.org}{PlanetMath}
and \href{https://github.com/planetmath}{PlanetMath github tex source files}.



\section{Introduction}

This project will begin with the translation of the Number Theory sections 
of PM (repository \href{https://github.com/planetmath/11_Number_theory}{github repository for PM Number Theory}).
(Future plans include the translation of  other PlanetMath repositories.)

The final product will consist of 
  many TeXa Colada files that pass through its parsing tools
  without error.

\subsection{File divisions}

%  Generally multiple PlanetMath files will be merged into larger Colada files, according to the first
%  three charcters of the MSC code (11A, 11B, 11C, etc.),
%  except tha 11A will use 5 digits;
%  and 11B and 11R  will use 4 digits, because of the
%  large number of articles in those categories.  (See the statistics below.)
 
 Generally, each PlanetMath file will correspond with a Colada file of the same name.
 
  \subsection{File Structure}
  File name conventions and directories should follow PlanetMath.  The topics
are given in CamelCase (no dashes are underscores), and the first
characters of the filename give the MSC subject classification.
The names of files should agree with 
\href{https://cran.r-project.org/web/classifications/MSC-2010.html}{MSC names}.
For example, MSC gives
\begin{verbatim}
11Mxx: Zeta and L-functions: analytic theory 
\end{verbatim}
This corresponds with a Colada file {\tt 11M-ZetaAndLFunctionsAnalyticTheory}.

%
\subsection{Missing Definitions} Definitions that are not supplied by PlanetMath can 
in rare cases be left as a
  incomplete stub.  A {\tt Fiat} file will record all incomplete definitions.  
  However, it is strongly preferred that these definitions
be inserted from another public source  such as Wikipedia with an external link.
Eventually the Fiat file should be empty.

A definition from another source should be placed in a separate
  section.

Definitions should always come from public sources. No paywalls and no registration should
block access the referenced definition source.  The use of copyrighted material should fall within 
licensed use, fair use, or
should be used in a transformative way to avoid copyright claims by outside parties.  

In choosing the source of an external definition, the long term public availability,
stability of links,
mathematical suitability (is this the definition a skilled mathematician would use?),
formalizability should be considered.


  
\subsection{Common Definitions}
 
A separate \emph{Foundations} file will provide many of the
  common core definitions of concepts such as sets, lists, finiteness,
  natural numbers, integers, rational numbers, real numbers, and so forth.

A separate \emph{Synonyms} file will provide many of the common
synonyms \emph{least upper bound/supremum/sup}, etc.   Specialized
synonyms should appear in the separate files.

Each word goes through a very crude \emph{singularization} process 
as it is tokenized.  (This algorithm can produce some strange results such as
singularizing \emph{series} to \emph{sery}, but in practice this is not a problem unless
an author tries to give distinct definitions to \emph{sery} and \emph{series}.)  
Words with the same singularization are treated as the same.  This means that
words such as \emph{number, numbers}, or \emph{zero, zeros}, or \emph{body, bodies} are
respectively treated the same and do not need explicit synonym declarations.   The singularization
algorithm does not consider the part of speech of the word, so that for example 
verbs are \emph{singularized}
just as readily \emph{intersects, intersect}, etc.
  
  Number theory definitions that get used regularly in other files should be
  moved to a common {\tt CommmonCore.tex} file that is imported by
  all.  This file will also contain common notation.
  


\subsection{Definitions from other MSC domains}

Some definitions outside number theory will be needed.  For example,
some definitions depend on definitions from general topology.  
Thus, as the project progresses, the project will spill into other areas, 
and other directories will be created that
contain common definitions from other MSC domains.  A file {\tt External.tex}
will record external dependencies on definitions.

\subsection{Linking} The package {\tt hyperref} should be used, and each section should
  link by a \verb!\href! back to the PlanetMath html document using
  the canonical name (for example,
  \href{https://planetmath.org/PoliteNumber}{PoliteNumber}).  Other PlanetMath metadata does not
  need to be included (we will generate this automatically).
  


\subsection{What to translate} 
Notation, definitions, and theorems should be translated.
Remarks and proofs should not  be translated.   However, the
remarks and proofs should remain in the Colada file (but confined
to remark sections, so that they are not parsed as Colada text).

The aim is to obtain Colada files that are as readable as the original
PlanetMath files.  Except for added definitions, the Colada files
should be nearly the same length as the PlanetMath files.
Any significant loss of readability or expansion in length should 
trigger discussion, and better solutions should be found.

\subsection{Logical Dependencies} 
Circularities and inequivalent redefinitions must be avoided.  
The Synonym, Foundation, Fiat, and CommonCore files will be
loaded first, then the separate
number theory files from beginning to end.  Files must be orderable so
that no definition is encountered before it is defined.

Definitions, notations and macros that appear in sections with local scope are by default not globally visible.
All synonyms are globally visible. 

Eventually, we expect to have a tool that will extract the declarations made and terms used in each file,
then reconstruct a dependency tree for all files in the directory.  Any volunteers?

\subsection{Dealing with unusual content} Some articles with unusual content can be omitted, such as 
 \href{11B99-196sReverseAndAddSequenceTo1000Terms.tex}{ReverseAndAddSequence}.  A file
  {\tt PlanetMathComments.tex} should describe significant departures
  between Planet Math and this project.
  Errors in the PlanetMath documents should be corrected in the Colada files
  and documented in {\tt PlanetMathErrata.tex} 
  (and
  they will be periodically forwarded to PlanetMath maintainers).
  Definitions and theorems that lack formal rigor should be reported in the Errata.  


\subsection{License}

The license for PlanetMath is Creative Commons Attribution-ShareAlike 3.0 Unported.
This project will be released under a compatible license.

\section{Statistics}

\begin{verbatim}

Number of articles in planetMath/11_Number_theory by MSC subject code.
$ ls *.tex | cut -c-3 | uniq -c
  73 11-
 468 11A
 118 11B
  18 11C
  36 11D
  23 11E
  12 11F
   9 11G
   7 11H
  33 11J
   5 11K
   7 11L
  31 11M
  50 11N
   7 11P
 138 11R
  12 11S
   5 11T
  10 11Y
   9 11Z
\end{verbatim}

\end{document}
