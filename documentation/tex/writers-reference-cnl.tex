\documentclass[12pt]{article}


% PACKAGES
\usepackage{url}
\usepackage{amsmath}
\usepackage{amsthm}
\usepackage{amssymb}
\usepackage{xcolor}
\usepackage{currfile}
\usepackage{fancyvrb}
\usepackage{xparse} % needed for \ellipsis control sequence in cnl-style
\usepackage{enumitem} % for topsep=0pt
\usepackage{colonequals}

% for underscores https://texfaq.org/FAQ-underscore
\usepackage{lmodern}
\usepackage[T1]{fontenc}
\usepackage{textcomp}
\usepackage{lineno}

\usepackage[
bookmarksopen,
bookmarksdepth=2,
%breaklinks=true
colorlinks=true,
urlcolor=blue]{hyperref}

% GLOBAL FORMATTING
%\linenumbers
\parindent=0pt
\parskip=0.5\baselineskip
\raggedbottom

% TITLE AUTHOR DATE
\title{Writer's Reference Manual for the Colada Language}

\date{November 1, 2019}
\author{Thomas Hales}

% THEOREMS
\newtheorem{definition}{Definition}
\newtheorem{theorem}[definition]{Theorem}
\newtheorem{lemma}[definition]{Lemma}
\numberwithin{definition}{section}
%\newtheorem{specification}[definition]{Specification}


% DOCUMENT

\begin{document}
\maketitle

\setcounter{tocdepth}{2}
\tableofcontents
\newpage

\newcommand{\Nat}{{\mathbb N}}
\newcommand{\Int}{{\mathbb Z}}
\newcommand{\Real}{{\mathbb R}}
\newcommand{\tmid}{//}

This is a writer's reference manual for the Colada language, as in Pina Colada.  The name
stands for {\bf CO}ntrolled {\bf LA}nguage {\bf DA}ta.

\section{Introduction}


By a controlled natural language for mathematics (CNL), we mean an
artificial language for the communication of mathematics that
\begin{itemize}
\item is deliberately designed with precise computer-readable syntax
  and semantics,
\item is based on a single natural language (which for us will be
  English),
\item and is broadly understood at least in an intuitive way by
  mathematically literate speakers of the natural language.
\end{itemize}




Here are some design goals.
\begin{itemize}
\item The language should narrow the gap between formalization and
  current mathematical written communication.
\item Writing documents in this language should be similar to writing
  \LaTeX.
\item Reading documents in this language should be similar to reading
  ordinary mathematical texts typeset in \LaTeX.  A short explanation
  should be all that is needed to explain the special conventions of
  the language.
\item The logical and mathematical foundations of the system are
  provided by the Calculus of Inductive Constructions (CiC) as
  implemented in the Lean theorem prover.  Documents written in the
  language should compile to CiC.
\end{itemize}


This is a manual for writing texts that have been written in the
Colada controlled natural language.  
We have tried to design a language that is rather close to current
conventions of written mathematical communication.  However, an exact
fit is not possible for various reasons.

\begin{itemize}
\item Most mathematical writing is imprecise.
\item The semantics of our language is dependent type theory.
\end{itemize}

It is easier to read a Colada document than to write one.%

%
This reference is meant to help mathematicians write the language.
A shorter separate document (the reader's guide to the Colada language)
gives a quick orientation to reading documents that have been
written in Colada.

As a first approximation, the Colada language can be viewed as the
disjoint union of the syntaxes of \LaTeX, dependent type theory, and
Forthel, glued together along a common semantic core.%
%
\footnote{The intended target semantics of the Colada language is the
  calculus of inductive constructions, as implemented in the Lean
  theorem prover.}
%
From a slightly simpler perspective, Colada documents are just
\LaTeX\ documents with precisely described semantics.  Colada documents are
generally written in \LaTeX, and this reader's guide describes how to
read the pdf document, as typeset with \LaTeX.

The language has been designed as a language in which mathematical
definitions and theorem statements can be expressed in a precise,
yet readable manner.   Stating something precisely in a computer
readable format is no guarantee of correctness.  The reader be warned!

From here on, it is assumed that the reader has already read the
reader's guide.

A separate guide also describes the process of writing \LaTeX\ documents
that are intended to be converted to Colada.  This reference assumes
that the reader has already read that document as well.

The Colada language was inspired by the Forthel language and its implementation
in Haskell in the Naproche-SAD repository. 

\section{Lexical Structure}

A Colada file is first converted to a stream of tokens or lexemes.
This section describes the lexical structure of the language.

The lexical structure can be understood without understanding the
full grammatical structure of the language.

The ASCII character set is used. Eventually, we will switch to unicode with
UTF8 encoding.

Comments are as in \TeX.  A comment begins with a \% symbol and continues to the
end of the line.  White space is ignored, except in establishing token boundaries.
White space is not needed between tokens, when boundaries can be established
without them
\[
{x,y,z} \quad\text{is equivalent to}\quad {\ x\ ,\ y\ ,\ z\ }
\]
The white space characters are the ASCII symbols for
space \verb!' '!, tab \verb!'\t'!, line feed \verb!'\n'!, and carriage return \verb!'\r'!.  
They are all interchangeable except
that only a carriage return ends a comment.

\subsection{delimiters}
The delimiters are
\[
(\quad )\quad [\quad ]\quad \{\quad \}.
\]
These characters cannot be combined in any way into a larger lexical unit.
Each opening delimiter must match with a closing delimiter of the same sort
in a correctly nested way.  There are no exceptions.

Parentheses $()$ are used for ordered pairs $(x,y)$ and tuples $(x,y,z)$.
They are used for type annotations $(x : \Nat)$, for grouping
expressions $x + (y+z)$, for cross-references (by theorem 3.4), for
describing the precedence of operators (with precedence 50 and left
associativity), among other purposes.

Square brackets $[]$ are used for lists $[1;2;3]$, and for instructions
[synonyms~integer/integers].

Braces $\{\}$ are used with control sequences \verb!\binomial{n}{k}!, multiple assignments
$\{\ x\ := 1;\ y\ := 2\ \}$, labeled arguments to functions, subtypes
$\{\ x : \Nat \tmid x > 3\}$, set enumerations $\{1,2,3\}$, set
comprehensions $\{x \mid x > 3\}$, and structure declarations.

\subsection{identifiers}

An alphanumeric character is any of a-z, A-Z, 0-9, \_, '.  

An identifier is a longest match of one or more alphanumeric
characters that is not part of a decimal number such as $3.14159$.
An identifier is tokenized as one of: BLANK, INTEGER, VAR,
WORD, ATOMIC\_IDENTIFIER.

If an identifier consists of a single \_, it is the BLANK token.
A blank can be used as a wild-card in matching statements, and
to mark a quantity that the author leaves to the computer to infer.

An INTEGER is an identifier consisting entirely of digits 0-9.

If an identifier consists of a single alphabetic character followed by
zero or more non-alphabetic characters, it is a VAR (variable) token.
Additionally (for purposes of mangling), a variable is a single
alphabetic character, followed by a double blank \_\_ and possibly additional
alphanumerical characters.  Examples of variables are
\[
x\quad x'\quad x\_1\quad A33\quad \text{V\_\_alpha}.
\]

A variable is used in the patterns in definitions.  In some contexts, a variable
may be used as the indefinite article 'a', as a word, or as an atomic
identifier.  

A WORD token is an identifier consisting of one or more alphabetic
characters that is not a variable.  An atomic identifier is an
identifier that is not a BLANK, VAR, INTEGER, or WORD.

Variables and atomic identifiers are case sensitive. Words are not.  (In a few
contexts words may be used as atomic identifiers, and in such contexts,
they are case sensitive.  For example, the trig function $\cos$ is
generally used as an atomic identifier, rather than an English word.)

When a word is first tokenized, it is converted to lower case and is
singularized.  Singularization roughly means that any final '-s' is removed.
More precisely, the following rules are applied (taking the first rule that
applies)
\begin{itemize}
\item Do not change a word not ending in 's'.
\item Do not change a word with at most $3$ characters.
\item Do not change a word if it would result in a word of $2$ characters or fewer.
\item Do not change a word ending in 'ss' or 'ous'.
\item Drop the final 'es' from a word ending in consonant + 'oes'.
\item Change the final 'ies' to 'y' in a word ending in consonant + 'ies'.
\item Drop the final 's' in words ending in 'izes' or 'ises'
\item Drop the final 'es' in words ending in 'ches', 'sses', 'shes', 'xes', or 'zes'
\item Otherwise, drop the final 's'. 
\end{itemize}

Singularization is applied to all words regardless of their part of
speech.  Words with the same singularization are treated as the same
word.  (Writers may introduce additional synonyms whenever
singularization fails to get it right.)  The point is that English
contains a great deal of redundant information that may be safely
ignored.  Correct agreement of subject and verb is done for the sake
of the reader, but not to assist the controlled natural language.

This simple algorithm is far from perfect, and sometimes the
singularization produces strange results such as 'series' becoming
'sery'.  Strange singular forms are generally harmless, as long as the
writer does not try to assign distinct meanings to 'series' and
'sery'.




\subsection{special characters}

A longest match rule applies in lexing.  

The punctuation marks are .\ ,\ ;\ .

The period is used to mark the end of statements.  

The period is used in hierarchical
identifiers (Math.cos), in field accessors ($G.\text{plus}$)
and in user-defined symbols $1..2$.  It is a decimal point $3.14159$.

Semicolons are used for list $[1;2;3]$, for multiple assignments 
$\{\ x\ := 1;\ y\ := 2\ \}$, and in place of the comma in certain complex statements.

The comma is used frequently in many contexts.  It is used between
bound variables and at the end of a bound variable block $\forall\ x,y,\ x + y = y + x$, 
and almost everywhere that a list of expressions is needed.



The colon's only use (as a separate token) is to make type annotations $(x : \Nat)$.


\section{Parsing}

\subsection{synonyms}

The syntax of a synonym declaration is
\[
\text{[synonyms dodecahedron/dodecahedra]}
\]
or when several synonyms are declared:

[synonyms analysis/analyses,
annulus/annuli,
appendix/appendices,
automaton/automata,
axis/axes,
calculus/calculi,
cicatrix/cicatrices,
continuum/continua
]
\]

If one of the synonyms adds a suffix to a word,
then an abbreviated style can be used.

[synonyms schema/schemata]

is equivalent to

[synonyms schema/-ta]

The synonym declaration can also be put into a sentence.

{\it We introduce synonyms matrix/matrices.}

Multiword synonyms are permitted.

[synonyms least upper bound/supremum/sup]

Phrases declared as synonyms become completely interchangeable in the
text.  Synonyms can be defined both for words and for case-sensitive
identifiers that have the form of a word (that is, consisting entirely
of alphabetic characters). For example, the functions $\acos$ and
$\arccos$ are treated as case-sensitive identifiers rather than words,
but we can still make them synonyms.

Synonyms cannot be formed with symbols.

[synonyms acos/arccos]

Synonym expansion occurs after singularization and after coarse parsing.
Synonym expansion occurs in a single linear pass beginning to end 
without overlap.  That is, the substituted text cannot be reused
as part of another synonym expansion.  For example.
if we declare synonyms

[synonyms AB CD EF/XY, CD/UV, EF GH/JK, XY GH/ST, GH/WX]

Then the phrase AB CD EF GH IJ is equivalent to XY GH and to XY WX,
because of the synonyms
AB CD EF/XY and GH/WX.  But it is not synonyms with
ST, nor with AB UV EF GH IJ.



Synonyms are implemented in an efficient way.  Synonyms have global scope.
Synonyms must be declared before the definition or macro using the words.
Synonyms cannot be redefined.  Synonyms cannot involve certain key words
of the grammar (such as {\it and}, {\it or}, {\it if}, and {\it the}.
They cannot begin with an even larger set of key words such as 
common articles and prepositions {\it a}, {\it of},
{\it in}, etc.

Synonyms cannot cross certain grammatical boundaries.
For example, assume we are in a context with a synonym instruction

[synonyms real number/reals]

and an adjective {\it totally real} and noun {\it number field}.
In the pharse {\it totally real number field}, the adjacent words {\it real number}
are not a candidate for synonym substitution with {\it reals}, because 
the two words are grammatically separate??









\end{document}

