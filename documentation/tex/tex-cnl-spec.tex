\documentclass[12pt]{amsart}


% PACKAGES
\usepackage{url}
\usepackage{amsmath}
\usepackage{amsthm}
\usepackage{amssymb}
\usepackage{fancyvrb}

% for underscores https://texfaq.org/FAQ-underscore
\usepackage{lmodern}
\usepackage[T1]{fontenc}

\usepackage{textcomp}


\usepackage[
bookmarksopen,
bookmarksdepth=2,
%breaklinks=true
colorlinks=true,
urlcolor=blue]{hyperref}

% GLOBAL FORMATTING
%\linenumbers
\parindent=0pt
\parskip=0.5\baselineskip
\raggedbottom

% TITLE AUTHOR DATE
\title{Notes on translating \TeX\ to a controlled natural language}

\date{July 17, 2019}                                           % Activate to display a given date or no date
\author{Thomas Hales}

% THEOREMS 
\newtheorem{definition}{Definition}
\newtheorem{theorem}[definition]{Theorem}
\newtheorem{lemma}[definition]{Lemma}
\newtheorem{specification}[definition]{Specification}

% COMMANDS
\renewcommand\labelitemi{-}


% ENVIRONMENTS

% \leavevmode\par is to make remark work when it is the first item in a subsection.
\newenvironment{remark}
{\leavevmode\par\begin{tabular}{|p{13cm}}\parskip=\baselineskip{\bf Remark.}}
{\end{tabular}}

\newenvironment{oblongo}{}{}

\newenvironment{prule}%
               {\begin{itemize}}%
               {\end{itemize}}
\newcommand{\ptem}{\item}
\newcommand{\nt}[1]{{\tt #1}}
\newcommand{\rw}{$\quad\to\quad$}



% DOCUMENT

\begin{document}
\maketitle



\section{\TeX-CNL translation specification}

The notes here add detail to the
 \href{https://github.com/formalabstracts/CNL-CIC/wiki/Conversion-from-LaTeX}{CNL-CIC Wiki}.

A \TeX\ file written in CNL-enabled \TeX\ can be processed 
by the \TeX\ engine or by the CNL engine.

%A boolean flag {\tt isPresentation} in the \TeX\ source determines
%which mode is used for processing.

When
we talk about the pdf display, we mean the pdf output 
from \TeX.

The CNL image
means the output of the \TeX\ to CNL translation.  The CNL image is a
text file written in the CNL grammar.


\subsection{Differences in CNL-enabled files}

There are some visible differences between a typical \TeX\ source
and a CNL-enabled one.

\subsection{Variables and Identifiers}

Identifiers can appear with the underscore character.  They must be
processed specially.   We use the fancy verbatim mode and surround
identifiers with ! as in \verb'!identifier_with_underscore!' .

Subscripts written without braces are interpreted as part of the variable
or identifier name.



Subscripts written with braces are interpreted in CNL as function applications.

Thus, \verb!x_1! is a variable and  \verb!x_{1}!  is the function $x$ applied to $1$.
This works with multiple indices $a_{i j}$ (with a space between the $i$ and the $j$),
is translated to CNL \verb!a \sb (i j)!  The function application gets suppressed between
the $i$ and the $j$, so that this is equivalent to \verb!a i j!.   To activate function application
in the subscript, wrap the subscripts in an extra layer of parentheses \verb!a_{(f x)}! or
\verb!a_{\parenI{f x}}!.  The \verb!\parenI! control sequence is invisible in the pdf display,
but expands to parentheses in the CNL.

\subsection{Words}

Words are generally formed by consecutive alphabetic characters.

When separate words are joined by \verb!\-!, \verb!\~!, or \verb!\'! they appear
as a dash, space, and apostrophe in the pdf, 
but they are removed in the CNL file.
For example, 
\verb!Riemann\-Roch! becomes \verb!RiemannRoch! in the CNL,
\verb!Wiener\'s\~theorem! becomes \verb!Wienerstheorem!,
\verb!totally\~bounded! becomes \verb!totallybounded!.

Alphabetic phrases enclosed in \verb!{~  }! become a single word in the CNL.
For example,
\verb!{~generalized Bernoulli shift}! becomes \verb!generalizedBernoullishift! in 
CNL, and
\verb!{~Sturm\-Liouville operators}! becomes \verb!SturmLiouvilleoperators!.

The sequences \verb!\~! and \verb!\'! are no longer available as accents.

Sometimes, one must choose whether to treat an adjective+noun as a
term with a left attribute, as in \verb!commutative ring! or as a
multi-word noun, as in \verb!normed\~ring!.  In this case, commutative
is made into an attribute because it is a predicate on ring. However,
`normed` adds data (the norm) to the ring, and it is treated as a
multi-word noun.



\subsubsection{labels}

Labels are converted to valid CNL identifiers by replacing
whitespace and non-alphanumeric characters to underscores.

\subsection{Special characters}

\subsubsection{superscripts}

\begin{itemize}
\item The use of the superscript mark \verb!^! should be restricted
to places where it acts as an infix (right associative) binary
operator for raising the term to a power.
\item When the superscript mark is used in connection with
subscripts as function call, the interpretation of \verb!f_i^n! is
the $n$th power of $f_i$, that is, $(f_i)^n$. 
Thus, this should be written in this
order rather than \verb!f^n_i! (which, would suggest the $n$th iterate
of $f$ applied to $i$).
\item the superscript character is treated like every other math symbol.
\item macros should be used when the meaning of a superscript is
not a power.
\end{itemize}

\subsubsection{invisible characters}

Invisible characters are generally given by macros
(with convention that their names end with $I$).  For example
\verb!\mulI! is the invisible multiplication operator.
\verb!x\mulI y! has pdf display $x y$, but the CNL image
preserves the binary operation \verb!x\mulI y!.

\subsubsection{math fonts} 

We treat variables in different fonts as different variables.  We achieve this by fusing the font into the variable name.
That is, when processing a source file containing \verb!\mathcal{C}!, it is not tokenized as a control sequence \verb!\mathcal! followed by
a begin group \verb!{!, etc.  Rather, it is tokenized as a single fused variable/identifier token \verb!\mathcal{C}!.

This is done at the lexical stage as follows.

Anything in the source file fitting the regular expression pattern
\begin{Verbatim}
%sedlex.regexp? "\\math", Star(alphabet), 
  lbrace, alphabet, Star(alphanum), rbrace, Star(alphanum) 
\end{Verbatim}
is treated as a single token.  This means the literal characters \verb!\math! followed by 0 or more alphabetic characters \verb!a-zA-Z!,
followed by a \verb!{!, another letter of the alphabet, 0 or more alphanumerics (the alphabet including \verb!_'0123456789!), \verb!}!, and then
terminated by 0 or more alphanumerics.   Thus, \verb!\mathcal{C}_1! and \verb!\mathfrak{g}'! match this regular expression.
No expansion is made of the control sequence starting with \verb!\math...!.  No white space is allowed.  

It will not generally work to use
a macro that expands to something
matching this regular expression.  Macro expansion takes place after the lexical stage is complete.

In particular, this pattern allows variable disambiguation based on the following math fonts (starting with \verb!\math...!):
\verb!\mathrm!, \verb!\mathit!, \verb!\mathnormal!, 
\verb!\mathcal!, \verb!\mathscr!, \verb!\mathpzc!, 
\verb!\mathbb!, \verb!\mathbbm!, \verb!\mathbbmss!, 
\verb!\mathbbmtt!, \verb!\mathds!, \verb!\mathfrak!.

Accents can be handled in the same way.  For example, a \verb!\mathtilde! macro
can be defined 

\verb!\def\mathtilde{\tilde}!

which in the CNL image will expand in the same fashion as a math font.



%https://tex.stackexchange.com/questions/58098/what-are-all-the-font-styles-i-can-use-in-math-mode


\subsection{control sequences}

\verb!\CNLcustom[k]\controlseq{pattern text}!   When translating \TeX\ to CNL use the pattern text to translate the control sequence \verb!\controlseq!
The integer $k$, which is optional, gives the number of braced arguments to read for expansion in the pattern text, after \verb!\controlseq!

For example, 
\verb!\CNLcustom[1]\parenI{ (#1) }! states that \verb!\parenI{X}! in the \TeX\ should translated to $(X)$ in the CNL.

For example,
\verb!\CNLcustom\oplus{ \vplus }!

A \verb!\noexpand! in the pattern text prevents the macro from being expanded, but the \verb!\noexpand! is not transcribed into the CNL image.

\verb!\CNLcustom\X{\noexpand\X}!.  Here \verb!\X! in the source becomes \verb!\X! in the image.

\verb!\CNLnoexpand[k]\controlseq! is equivalent to 
\begin{Verbatim}
     \CNLcustom[k]\controlseq{\noexpand\controlseq}
\end{Verbatim}

\verb!\CNLdelete[k]\controlseq!  When translating to CNL, delete the control sequence \verb!\controlseq! together with $k$ curly braced arguments.
This is equivalent to giving empty expansion text \verb!\CNLcustom[k]\controlseq{}! 

\verb!\CNLexpand[k]\controlseq! expands the macro with $k$ arguments, but using the macro definition rather than \verb!\CNLcustom! pattern text.


\subsubsection{lists and curried functions}

Unlike in the source, in the CNL functions are generally curried.
In the CNL, lists are generally demarcated with square brackets and semicolons.


\verb!\app{f}{x1,....,xN}! (which displays as $f(x1,...,xN)$) becomes curried in CNL \verb!((f) (x1) (x2) ... (xN))!

\verb!\list{x1,...,xN}! (which displays as $x1,\ldots,xN$)
becomes a list in the CNL image \verb![x1;...;xN]!  (Here the $x_i$s must have the same type.)

For \TeX\ display, the list can be wrapped in delimiters to form tuples, set enumerations, etc.
For example, 
\begin{Verbatim}
\def\tuple#1{(#1)}
\def\braced#1{\{#1\}}
\end{Verbatim}

For the CNL image, we may generate lists, set enumerations, etc. 
\begin{Verbatim}[fontfamily=tt,showspaces=false]
\CNLcustom[1]\tuple{\list{#1}}
\tuple{1,2,3} % CNLimage: [1;2;3]
              % source expansion {1,2,3}

\CNLcustom[1]\braced{\setenum{\list{#1}}} 
\braced{1,2,3} % CNLimage: \setenum{[1;2;3]}
               % source expansion: \{1,2,3\}
\end{Verbatim}


\subsubsection{stripped characters}

In translating to the CNL image, 
many formatting characters are removed from the \TeX\ source files.

This include instructions to change among different
\TeX\ modes:  

\verb!$ \( \) \begin{math} \end{math}!, 

\verb!\[ \] $$ \begin{displaymath} \end{displaymath}!.

Rules, glues, and space \verb!\  \, \medskip, \quad!, etc.

\subsubsection{curly braces}

Some curly braces are stripped.  Some are converted to
parentheses.  Here is the rule.  Curly braces marking a
subscript are replaced by parentheses.  Curly braces used
to demarcate control sequence arguments are stripped.



\subsubsection{ambiguous notations}

Lean elaboration does substantial work to resolve
notational ambiguities.  Some of these ambiguities
may be resolved directly by the author in the
\TeX\ source files, with notations that appear identical
in the pdf display.

For example,
\begin{itemize}
\item \verb!\closedInterval{a}{b}! -\qquad $[a,b]$
\item \verb!\LieCommutator{a}{b}! -\qquad $[a,b]$
\item \verb!\groupCommutator{a}{b}! -\qquad $[a,b]$
\item \verb!\image{f}{S}! -\qquad $f(S)$
\item \verb!f(S)! -\qquad $f(S)$
\item \verb!\fieldDegree{K}{F}! -\qquad $[K:F]$
\item \verb!\subgroupIndex{G}{H}! -\qquad $[G:H]$
\end{itemize}

\subsubsection{foreign accents}

In material originating in \TeX horizontal mode, 
the CNL image is
stripped of many of the foreign accents.

\begin{itemize}
\item G\"odel - Godel
\item \'etale - etale
\end{itemize}

The \verb!\~! has been redefined to be a space (that disappears in the CNL file).

\subsubsection{control sequences as variables}

We wish for it to be possible to use some control sequences (and more general token lists)
as variables: \verb!\alpha!, etc.
We use special control sequences \verb!\var! and \verb!\id! for this.
They have no effect on the \TeX:

\verb!\def\var#1{#1}!

\verb!\def\id#1{#1}!

\verb!\var{\alpha_1} --> V__alpha_1!. In the CNL, \verb!V__! is always prepended.  Nonalphanumeric
characters are deleted.

\verb!\id{\alpha_1} --> id_alpha_1!.  In the CNL, \verb!id_! is always prepended. Nonalphanumeric
characters are deleted.

It is up to the user to avoid name collisions in the CNL image.

No macros are expanded inside \verb!\id! and \verb!\var!  That is, these macros freeze what is in their
replacement patterns.   
For example,

\verb!\var{\mathcal{C}} --> V__mathcalC!

But in the other direction, other macros can have replacement patterns that
expand to \verb!\id! and \verb!\var!:

\verb!\def\myalph#1{\var{\alpha_{#1}}}!.

\verb!\myalpha{33} --> V__alpha_33!.


\section{environments}

All text is entirely ignored outside the
\verb!\begin{cnl}...\end{cnl}! environment.

Environments should be properly nested.  In particular, they should be properly
nested and terminated within the cnl environment.

\verb!\input! should not be enclosed within any environment, except possibly a \verb!\begin{cnl}! environment.
The input file is read iff the input statement is enclosed in the cnl environment.

In CNL mode, there is a list of ignored environments, such as the
\verb!remark! environment.  No macro expansion is performed in
the ignored environments.  However, bracketing must be properly nested.


In CNL mode, there is a list of translated environments, with custom
translation for each of them, such as the \verb!definition!  and
\verb!theorem! environments.


\subsection{specific environments}

Here we describe what happens to the \TeX sources, 
upon translation to CNL-image.

% https://latex.wikia.org/wiki/List_of_LaTeX_environments
% http://www.personal.ceu.hu/tex/environ.htm

\subsubsection{definition environment}

Theorems, Definitions, etc. can carry labels.  In fact, labels are
encouraged.  The labels should be valid atomic identifiers.  In the
CNL image, these labels become CNL labels.

Theorems, etc. can also carry optional names

\verb!\begin{theorem}[optional_name]\label{optional_name}!

When a name is given, the label should also appear, and the two should
be the same.


\subsubsection{paragraph environments}

\begin{description}
\item [center] Environment begin/end stripped, leaving enclosed text.
\item [flushleft, flushright, minipage, quotation, quote, verse]  begin/end as well as formatting options are stripped, leaving enclosed text.
\item [figure, picture, remark, thebibliography, titlepage] Environment begin/end and enclosed text stripped.
\item [tabbing]  begin/end stripped as well as tabbing characters \verb!\= \> \+ \-!
\item [tabular] not supported.
\end{description}

\subsubsection{equation alignment environments}

\begin{description}
\item [array] The begin/end and column format
instructions are stripped.  The array is converted
to a matrix (list of lists) in the CNL image.
\item [align, eqnarray, gather] begin/end and \verb!&! stripped.
A list is created with separator \verb!\\! translated
to separator \verb!;!
\item [equation,equation*]
begin/end and \verb!&! stripped.
Translation to CNL-Equation.  Any label is translated.
\item [multline, split]  begin/end and \verb!& \\! stripped.
\item [flalign, flalign*]  Not supported.
\end{description}

See also gathered, split, aligned alignedat, alignat* 

\subsubsection{matrix environments}

\begin{description}
\item [matrix, pmatrix, bmatrix, Bmatrix, vmatrix, Vmatrix, smallmatrix]   begin/end is stripped. The data is converted to a list of lists.  The list of lists is
given as an argument to a control sequence with the same name as the environment \verb!\pmatrix{[[0;1];[2;3]]}!
\item [cases] Not supported.
\end{description}



\subsection{itemize environment}

This includes itemize, enumerate, description.

Many objects can be itemized in the CNL.  We create
custom itemize environments for them.
This includes special \TeX\ support for match, function matching,
cases, inductive
and mutual inductive type declarations, structures, etc.
These custom itemize environments are translated to the
appropriate language constructs in the CNL image.

\section{naming conventions for macros}

The successful implementation of the the \TeX\ to CNL translation
will rely on a very large number of macros.  In an ordinary \TeX\ document,
most macro names are largely invisible to everyone except the author.
However, in our system,
macro names become part of a CNL document and part of
the formal specification of a mathematical definition.  Therefore, increased
care is called for in choosing macro names.


We suggest some naming conventions.
We recommend
\href{https://hilton.org.uk/presentations/naming-guidelines}{naming-conventions}.
Several of our rules have been adapted directly from that source.


Follow \LaTeX\ 
\href{https://tex.stackexchange.com/questions/48195/macro-naming-best-practice}{capitalization conventions}.  Most document macros are all lower case.
(Common exceptions are upper case alphabets \verb!\Delta! \verb!\AA! (\AA), 
proper names,  
and variants of a lower case macro
such as \verb!\rightarrow! and \verb!\Rightarrow!.)
Package interface commands are CamelCase, and internal commands contain
the symbol \verb!@!.  
We will be tolerant of allowing camelCase (starting with lowercase) in 
document-level macros.  When using camelCase, upper-case characters
should always mark the beginning of a word in the name.

Where \LaTeX\ has established a precedent for naming macros, follow that.

The order of words inside a name should follow the order of hierarchical identifiers for
namespaces.  The real absolute value would probably be found in the real numbers namespace
and the absolute value on integers would probably be found in the integer namespace.
As hierarchical identifiers for namespaces, 
we would have \verb!real.abs! and \verb!int.abs!.  
Thus, \verb!\realabs! and \verb!\intabs! should be the macro names
(rather than say, \verb!\absreal!). 

Complete correctly-spelled words are better than abbreviations.  
Abbreviations should be dictionary-accepted abbreviations, or widely recognized
math abbreviations (such as int for integer, nat for natural number, bool for
boolean).  

It is expected that editor tools for
word completion will be used, and the user has a better chance of guessing
part of the macro name if the full word is used.  Distinct meanings call for
distinct names.  The names should reflect the semantics.




\begin{tabular}{l l l}
 \verb!\QuaternionRing! for ${\mathbb H}$ \\
 \verb!\Diracdelta! for $\delta$ \\
 \verb!\Kroneckerdelta! for $\delta$ \\
 \verb!\Feigenbaumdelta! for $\delta$
\end{tabular}



Following entropy, frequently
used macros may have shortened names.  Follow Knuth's \TeX\ style in this regard.
 
\begin{tabular}{ l l l }
common & \verb!\le! \verb!\it! \verb!\ne! \verb!\min! \\
uncommon & \verb!\displaywidowpenalty! \verb!\finalhyphendemerits!
\end{tabular}

Limit name length to at most 20 characters, and limit the word count in the
name to 4.  Generally avoid plural nouns in names. 


Name qualifiers should be suffixes (not prefixes). For example,
\verb!\sumBig! and \verb!\sumSmall!

Do not redefine existing macros.

Make the macro vocabulary large and mathematically precise.
Let entropy, the theory of error correcting codes, and searchability 
be guiding principles in macro naming. 
That is, macro names should be easy to find by search tools, and it
should be difficult to mistake one macro for another.  A mathematician
who knows the mathematical name of something should be able to
find the relevant macro by keyword search.

Different macros should differ in more than one or two letters.
\LaTeX\ violates this in places, such as \verb!\ldots! and \verb!\cdots!.
But avoid creating further examples like this.

Different macros should differ in more than word order.
We make an exception when the ordering aligns with the semantics.

\begin{tabular}{lll}
 incorrect: &\verb!\sumfromto{a}{b}{f}!\\ &\verb!\sumtofrom{a}{b}{f}!\\
 permissible: &\verb!\realintervalopenclosed{a}{b}! for $(a,b]$\\
    &\verb!\realintervalclosedopen{a}{b}! for $[a,b)$.
\end{tabular} 

Users can always define their own abbreviated macro names for local use.
Those will not become part of the shared libraries.


\section{outstanding issues}



\subsubsection{curly braces and font selection}



We generally wish to strip font variations \verb!{\it such as italics}! in
text mode.  Note that this doesn't fit the usual pattern of control sequence applied
to arguments.




\subsubsection{ellipsis}

There should be a general construction 

\begin{equation}\label{ellipsis}
\verb!I \op \dots \op J!   
\end{equation}

that strips $I$ and $J$ to the distinct core (taking the two token sequences and successively removing identical tokens from
both ends), which should be two integer expressions $i$ and $j$.
So that $I$ is the $\beta$-reduction of $K\ i$ and $J$ is the $\beta$-reduction of $K\ j$, for some
term $K$.  $K$ is the token sequence  \verb!(\LAM v, sI)! where $v$ is a fresh variable and $sI$ is the token sequence
obtained by replacing the distinct core of $I$ with $\verb!v!$.
Then \eqref{ellipsis} represents (for associative operation \verb!\op!).

\[
\verb! (H\ i) \op (H\ {i+1}) \op (H\ {i+2}) ... \op (H\ j)!
\] 

For example, in
\begin{equation}\label{W}
\verb!W =\{\omega_{0}(\bu),\ldots,\omega_{3}(\bu)\}!
\end{equation}
we have \verb!I=\omega_{0}(\bu)!, \verb!J=\omega_{3}(\bu)!, $i=0$, $j=3$, 

We can manufacture a \TeX\ macro \verb!\ellipsis#1#2#3#4!
Where the arguments are $H$ (as a term without the binder), $i$, $j$, \verb!\op!, respectively
that constructs the correct \TeX'ed pdf.  Roughly, $H$ should contain say a \verb!\DUMMY! where
the insertion is to take place.  The insertions are handled, say with something of the form \verb!\futurelet\DUMMY\CH i!,
where \verb!\CH! expands to $H$.

We can add custom rendering to the CNL image
by binding the term $H$ as \verb!(\LAM \DUMMY, H)! and applying it to $i$ and $j$.  

This would not be hygienic; we would not be able to nest ellipses.

For example, in \eqref{W}, we would write
\[
\verb!W = \{ \ellipsis{\omega_{\DUMMY}(\bu)}{0}{3}{,} \}!.
\]



\section{References}

\href{https://en.wikibooks.org/wiki/LaTeX/Mathematics}{Wikibook LaTeX Mathematics}

\href{https://hilton.org.uk/presentations/naming-guidelines}{naming-conventions}

\href{https://www.tug.org/TUGboat/tb32-3/tb102verna.pdf}%
{Towards \LaTeX\ coding standards}


\end{document}
